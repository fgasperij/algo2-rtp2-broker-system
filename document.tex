\documentclass[a4paper,10pt]{article}
\usepackage[paper=a4paper, hmargin=1.5cm, bottom=1.5cm, top=3.5cm]{geometry}
\usepackage[utf8]{inputenc}
\usepackage[T1]{fontenc}
\usepackage[spanish]{babel}
\usepackage{xspace}
\usepackage{xargs}
\usepackage{ifthen}
\usepackage{aed2-tad,aed2-symb,aed2-itef}
\usepackage{fancyhdr}
\usepackage{latexsym}
\usepackage{lastpage}
\usepackage[colorlinks=true, linkcolor=blue]{hyperref}
\usepackage{calc}
%El siguiente paquete permite escribir la caratula facilmente
\usepackage{caratula}
\usepackage{algorithm2e}


\newcommand{\moduloNombre}[1]{\textbf{#1}}

\let\NombreFuncion=\textsc
\let\TipoVariable=\texttt
\let\ModificadorArgumento=\textbf
\newcommand{\res}{$res$\xspace}
\newcommand{\tab}{\hspace*{7mm}}

\newcommandx{\TipoFuncion}[3]{%
  \NombreFuncion{#1}(#2) \ifx#3\empty\else $\to$ \res\,: \TipoVariable{#3}\fi%
}
\newcommand{\In}[2]{\ModificadorArgumento{in} \ensuremath{#1}\,: \TipoVariable{#2}\xspace}
\newcommand{\Out}[2]{\ModificadorArgumento{out} \ensuremath{#1}\,: \TipoVariable{#2}\xspace}
\newcommand{\Inout}[2]{\ModificadorArgumento{in/out} \ensuremath{#1}\,: \TipoVariable{#2}\xspace}
\newcommand{\Aplicar}[2]{\NombreFuncion{#1}(#2)}

\newlength{\IntFuncionLengthA}
\newlength{\IntFuncionLengthB}
\newlength{\IntFuncionLengthC}
%InterfazFuncion(nombre, argumentos, valor retorno, precondicion, postcondicion, complejidad, descripcion, aliasing)
\newcommandx{\InterfazFuncion}[9][4=true,6,7,8,9]{%
  \hangindent=\parindent
  \TipoFuncion{#1}{#2}{#3}\\%
  \textbf{Pre} $\equiv$ \{#4\}\\%
  \textbf{Post} $\equiv$ \{#5\}%
  \ifx#6\empty\else\\\textbf{Complejidad:} #6\fi%
  \ifx#7\empty\else\\\textbf{Descripción:} #7\fi%
  \ifx#8\empty\else\\\textbf{Aliasing:} #8\fi%
  \ifx#9\empty\else\\\textbf{Requiere:} #9\fi%
}

\newenvironment{Interfaz}{%
  \parskip=2ex%
  \noindent\textbf{\Large Interfaz}%
  \par%
}{}

\newenvironment{Representacion}{%
  \vspace*{2ex}%
  \noindent\textbf{\Large Representación}%
  \vspace*{2ex}%
}{}

\newenvironment{Algoritmos}{%
  \vspace*{2ex}%
  \noindent\textbf{\Large Algoritmos}%
  \vspace*{2ex}%
}{}


\newcommand{\Titulo}[1]{
  \vspace*{1ex}\par\noindent\textbf{\large #1}\par
}
%DiccTrie
\newcommand{\Title}[1]{
  \vspace*{1ex}\par\noindent\textbf{\large #1}\par
}

\newenvironmentx{Estructura}[2][2={estr}]{%
  \par\vspace*{2ex}%
  \TipoVariable{#1} \textbf{se representa con} \TipoVariable{#2}%
  \par\vspace*{1ex}%
}{%
  \par\vspace*{2ex}%
}%

\newboolean{EstructuraHayItems}
\newlength{\lenTupla}
\newenvironmentx{Tupla}[1][1={estr}]{%
    \settowidth{\lenTupla}{\hspace*{3mm}donde \TipoVariable{#1} es \TipoVariable{tupla}$($}%
    \addtolength{\lenTupla}{\parindent}%
    \hspace*{3mm}donde \TipoVariable{#1} es \TipoVariable{tupla}$($%
    \begin{minipage}[t]{\linewidth-\lenTupla}%
    \setboolean{EstructuraHayItems}{false}%
}{%
    $)$%
    \end{minipage}
}

\newcommandx{\tupItem}[3][1={\ }]{%
    %\hspace*{3mm}%
    \ifthenelse{\boolean{EstructuraHayItems}}{%
        ,#1%
    }{}%
    \emph{#2}: \TipoVariable{#3}%
    \setboolean{EstructuraHayItems}{true}%
}

\newcommandx{\RepFc}[3][1={estr},2={e}]{%
  \tadOperacion{Rep}{#1}{bool}{}%
  \tadAxioma{Rep($#2$)}{#3}%
}%

\newcommandx{\Rep}[3][1={estr},2={e}]{%
  \tadOperacion{Rep}{#1}{bool}{}%
  \tadAxioma{Rep($#2$)}{true \ssi #3}%
}%

\newcommandx{\Abs}[5][1={estr},3={e}]{%
  \tadOperacion{Abs}{#1/#3}{#2}{Rep($#3$)}%
  \settominwidth{\hangindent}{Abs($#3$) \igobs #4: #2 $\mid$ }%
  \addtolength{\hangindent}{\parindent}%
  Abs($#3$) \igobs #4: #2 $\mid$ #5%
}%

\newcommandx{\AbsFc}[4][1={estr},3={e}]{%
  \tadOperacion{Abs}{#1/#3}{#2}{Rep($#3$)}%
  \tadAxioma{Abs($#3$)}{#4}%
}%


\newcommand{\DRef}{\ensuremath{\rightarrow}}

%% CARATULA %%
%El siguiente paquete permite escribir la caratula facilmente
%\usepackage{caratula}
\hypersetup{
  pdftitle={ Algoritmos y estructuras de datos II - RTP2 },
  colorlinks,
  citecolor=black,
  filecolor=black,
  linkcolor=black,
  urlcolor=black 
}

%Datos para la caratula
\materia{Algoritmos y Estructuras de Datos II}

\titulo{RTP 2 - Broker System}

%\fecha{DDDD dd de MMMM, AAAA}

\grupo{Grupo 20}

\integrante{Fernando Gasperi Jabalera}{56/09}{fgasperijabalera@gmail.com}
\integrante{Esteban Romero}{659/06}{estebantaborcias@gmail.com}
\integrante{Leandro Tozzi}{-}{leandro.tozzi@gmail.com}
\integrante{Alfredo Terrile Cendoya}{022/11}{freddy199\_0@hotmail.com}

\parskip=5pt % 10pt es el tamaño de fuente

% Pongo en 0 la distancia extra entre ?temes.
\let\olditemize\itemize
\def\itemize{\olditemize\itemsep=0pt}

% Acomodo fancyhdr.
\pagestyle{fancy}
\thispagestyle{fancy}
\addtolength{\headheight}{1pt}
\lhead{Algoritmos y Estructuras de Datos II}
\rhead{$1^{\mathrm{er}}$ cuatrimestre de 2014}
\cfoot{\thepage /\pageref{LastPage}}
\renewcommand{\footrulewidth}{0.4pt}

\author{Algoritmos y Estructuras de Datos II, DC, UBA.}
\date{}
\title{RTP 2: Diseño \\ Broker System \\ Grupo: 20}



\begin{document}

\thispagestyle{empty}

\maketitle
\tableofcontents

\newpage
\section{Observaciones generales}

Convenciones que adoptamos en todos los móduos:
\begin{itemize}
	\item nos referimos a los campos de las tuplas por el nombre de los mismos, no por $\prod_1$, $\prod_2$,...,$\prod_n$.
    \item en los algoritmos utilizamos los alias de los tipos de tuplas que definimos en la estructura de representación. Por ejemplo, si en la estructura de representación definimos una tupla que la llamamos tuplaEspecial:\\
    \begin{Tupla}[tuplaEspecial]
   	  \tupItem{campoEspecial$_1$}{tipo$_1$}%
      \tupItem{campoEspecial$_2$}{tipo$_2$}%
      \tupItem{campoEspecial$_3$}{tipo$_3$}%            
    \end{Tupla}\\ 
después en los algoritmos cada vez que usemos una tupla con esos tipos usamos el alias tuplaEspecial y nos referimos a sus campos por campoEspecial$_1$, campoEspecial$_2$ y campoEspecial$_3$.
\end{itemize}

\textbf{Correcciones realizadas:}\\
\\
Diccionario Clientes
\begin{itemize}
	\item (estructura) d.claves dejó de ser un arregloOrdenado para ser un arregloDimensionable.
	\item (estructura) agregamos un dc.tamanio que nos dice cuántas claves hay definidas en el diccionario. Esto es necesario porque para asegurar que obtener toma O(log n) necesitamos saber exactamente dónde están las claves en el arreglo. En este caso estarán en las primeras [0, dc.tamanio) posiciones.
	\item (invariante) agregamos la condición de que d.claves debe estar ordenado.
	\item (invariante) agregamos el invariante escrito en lenguaje coloquial.
	\item (algoritmos) la búsqueda binaria se realiza en el mismo obtener del diccionario.
\end{itemize}
Diccionario Títulos
\begin{itemize}
	\item (estructura) sacamos la lista de significados del diccionario.
	\item (estructura) cambiamos la estructura del iterador porque ahora sólo guardamos una lista de claves y la de significados no.
	\item (invariante) agregamos la restricción de que dos padres no pueden compartir un hijo.
	\item (invariante) hicimos las modificaciones necesarias para que la nueva estructura de representación fuera consistente.
	\item (algoritmos) corregimos el error de agregar claves repetidas al listado de claves lo cual hacía que el iterador recorra repetidos y las complejidades no se satisfacieran.
	\item (algoritmos) el agregarAtrás(dc.claves, c) solo lo hacemos en los casos en los que la clave no estaba previamente definida.
	\item (algoritmos) modificamos todos los algoritmos del iterador para que operen sobre la nueva estructura.
	\item corregimos un error en la axiomatización de ObtDeEstruct, función auxiliar de la función de abstracción, que consideraba el caso en el que no estaba definida la clave que recibía y nunca era llamada de esa forma.
\end{itemize}

\section{Módulo Wolfie}


\begin{Interfaz}
  
  \textbf{parámetros formales}\hangindent=2\parindent\\
  \parbox{1.7cm}{\textbf{géneros}} $\alpha$\\
  \parbox[t]{1.7cm}{\textbf{función}}\parbox[t]{\textwidth-2\parindent-1.7cm}{%
    \InterfazFuncion{Copiar}{\In{a}{$\alpha$}}{$\alpha$}
    {$res \igobs a$}
    [$\Theta(copy(a))$]
    [función de copia de $\alpha$'s]
  }

  \textbf{se explica con}: \tadNombre{Wolfie}.

  \textbf{géneros}: \TipoVariable{wolfie}.

  \textbf{Operaciones básicas de Wolfie}

  \InterfazFuncion{InaugurarWolfie}{\In{clientes}{conj(clientes)}}{wolfie}
  [$\neg \emptyset ?(clientes)$]
  {$res \igobs inaugurarWolfie(clientes)$}%
  [$O(\#(clientes)^2)$]
  [genera un wolfie con los clientes recibidos en $clientes$.]

  \InterfazFuncion{AgregarTítulo}{\Inout{w}{wolfie}, \In{nomTit}{string}, \In{maxAcciones}{nat}, \In{cot}{nat}}{}
  [$w \igobs w_0$ $\land$  ($\forall$ $t$ : $titulo$) ($t \in$ títulos($w_0$) $\Rightarrow$ nombre($t$) $\neq$ nomTit) ]
  {$w \igobs$ agregarTítulo(crearTítulo($nomTit$, $cot$, $maxAcciones$), $w_0$)}
  [$\Theta(copy(a))$]
  [agrega un título a $w$ con el nombre $nt$, la cotización $cotizacion$ y un tope máximo de acciones $maxAcciones$.]
  
  \InterfazFuncion{ActualizarCotización}{\Inout{w}{wolfie}, \In{nomTit}{string}, \In{cot}{nat}}{}
  [$w \igobs w_0$ $\land$ ($\exists$ $t$:$titulo$) ($t \in$ títulos($w_0$) $\land$ nombre($t$) $=$ nomTit)]
  {$w \igobs$ actualizarCotización($nomTit$, $cot$, $w_0$)}
  [$\Theta(copy(a))$]
  [actualiza la cotización del título cuyo nombre es $nt$ a la cotización $cotizacion$ y ejecuta las promesas de venta y compra que puedan hacerse dada la nueva cotización del título.]
  
  \InterfazFuncion{AgregarPromesa}{\Inout{w}{wolfie}, \In{cliente}{cliente}, \In{nomTit}{string}, \In{tipo}{string}, \In{umbral}{nat}, \In{cantidad}{nat}}{}
  [$w \igobs w_0$ $\land$ ($\exists$ t:título) ($t$ $\in$ títulos($w_0$)
  $\land$ nombre($t$) = nomTit) $\land$ $c$ $\in$ clientes($w_0$) $\yluego$
  ($\forall p:promesa$)($p \in$ promesasDe($c$, $w_0$) $\Rightarrow$
  ($nomTit$ $\neq$ título($p$) $\lor$ $tipo$ $\neq$ tipo($p$)) $\land$ (tipo = vender $\Rightarrow$ accionesPorCliente($c$, título($p$)) $\geq$ cantidad($p$)))
  ]
  {$w \igobs$ agregarPromesa($c$, crearPromesa($nomTit$, $tipo$, $umbral$, $cantidad$), $w_0$)}
  [$\Theta(copy(a))$]
  [agrega una promesa de tipo $tipo$ al cliente $c$ sobre el título cuyo nombre sea $nomTit$.]    
  
  \InterfazFuncion{clientes}{\In{w}{wolfie}}{itDiccClientes(nat, infoCliente)}
  [$true$]
  {haySiguiente($res$) $\land$ esPermutación(SecuSuby($res$), clientes($w$))}
  [$\Theta(copy(a))$]
  [devuelva un iterador a el diccionario de clientes.]
  
  \InterfazFuncion{títulos}{\In{w}{wolfie}}{itDiccTítulos(string, infoTitulo)}
  [$true$]
  {haySiguiente($res$) $\land$ esPermutación(SecuSuby($res$), títulos($w$))}
  [$\Theta(copy(a))$]
  [devuelva un iterador a el diccionario de títulos.]
  
  \InterfazFuncion{promesasDe}{\In{w}{wolfie}, \In{c}{cliente}}{itLst(promesasTítulo)}
  [$c$ $\in$ clientes($w$)]
  {$res \igobs$ promesasDe($c$, $w$)}
  [$\Theta(copy(a))$]
  [devuelve todas las promesas del cliente $c$.]
  
  \InterfazFuncion{accionesPorCliente}{\In{w}{wolfie}, \In{nomTit}{string}, \In{cliente}{c}}{nat}
  [$c$ $\in$ clientes($w$) $\land$ ($\exists t:t\'itulo$)($t \in$ títulos($w$) $\land$ nombre($t$) = $nomTit$)]
  {$res \igobs$ accionesPorCliente($c$, $nomTit$, $w$)}
  [$\Theta(copy(a))$]
  [devuelve la cantidad de acciones que tiene el cliente $c$ del título cuyo nombre es $nomTit$.]
  
  \InterfazFuncion{enAlza}{\In{w}{wolfie}, \In{nomTit}{string}}{bool}
  [($\exists t:título$)($t \in$ títulos($w$) $\land$ nombre($t$) = $nomTit$)]
  {$res \igobs$ enAlza($nomTit$, $w$)}
  [$\Theta(copy(a))$]
  [devuelve $true$ si el título acaba de agregarse a Wolfie o si la cotización actual es mayor a la anterior.]
 
\end{Interfaz}

\begin{Representacion}
  
  \textbf{Representación de Wolfie}

  \begin{Estructura}{$wolfie$}[wolfieEstr]
    \begin{Tupla}[wolfieEstr]
      \tupItem{clientes}{DiccionarioClientes(cliente, infoCliente)}%
      \tupItem{títulos}{DiccionarioTítulos(nombre, infoTítulo)}%
      \tupItem{promesasDe}{infoPromesas}%
    \end{Tupla}
    
    \begin{Tupla}[infoPromesas]
      \tupItem{cliente}{nat}%
      \tupItem{actualizado}{bool}%
      \tupItem{promesas}{lst(promesaTítulo)}%      
    \end{Tupla}
    
    \begin{Tupla}[promesaTítulo]
      \tupItem{nomTit}{string}%
      \tupItem{tipo}{string}%
      \tupItem{umbral}{nat}%
      \tupItem{cantidad}{nat}%
    \end{Tupla}
    
    \begin{Tupla}[infoTítulo]
      \tupItem{maxAcciones}{nat}%
      \tupItem{accionesDisponibles}{nat}%
      \tupItem{cotización}{nat}%
      \tupItem{enAlza}{bool}%
      \tupItem{rachaMaxima}{nat}%
      \tupItem{rachaActual}{nat}%
      \tupItem{fluctuaciones}{nat}%
    \end{Tupla}

    \begin{Tupla}[infoCliente]    
      \tupItem{títulos}{DiccionarioTítulos(nombre, infoTítuloCliente)}
      \tupItem{totalAcciones}{nat}%      
    \end{Tupla}
    
    \begin{Tupla}[infoTítuloCliente]
      \tupItem{cantidadAcciones}{nat}%
      \tupItem{promesas}{promesas}%
    \end{Tupla}
    
    \begin{Tupla}[promesas]      
      \tupItem{compra}{promesa}
      \tupItem{venta}{promesa}%            
    \end{Tupla}
    
    \begin{Tupla}[promesa]
   	  \tupItem{pendiente}{bool}%
      \tupItem{umbral}{nat}%
      \tupItem{cantidad}{nat}%            
    \end{Tupla}
    
    \par donde cliente es nat
        
  \end{Estructura}
  \textbf{Invariante de representación}\\* \\* Entre wolfieEstr.clientes y wolfieEstr.Títulos:
 \begin{enumerate}
 	\item Todos los títulos que están definidos en los infoCliente.títulos también están definidos en el wolfieEstr.títulos.\\
    ($\forall claveCliente:cliente$) (def?($claveCliente$, $w$.$clientes$) $\Rightarrow$ ($\forall nomTit:titulo$)\\ (def?($nomTit$, dameTítulos($claveCliente$, $w$.$clientes$)) $\Rightarrow$ def?($nomTit$, $w$.$titulos$)))
    \item Para cada título definido en wolfieEstr.títulos el infoTítulo.accionesDisponibles es igual a la resta entre: infoTítulo.maxAcciones y la suma de la cantidad de acciones de ese título que tienen todos los clientes, es decir, la suma de los infoTítuloCliente.cantidadAcciones que se correspondan con el nombre del título que estamos calculando de todos los clientes.\\
    ($\forall nomTit:string$) (def?($nomTit$, $w$.$titulos$) $\Rightarrow$ dameDisponibles($nomTit$, $w$.$titulos$) = \\dameMaxAcciones($nomTit$, $w$.$titulos$) - sumatoriaAccionesTítulo($t$, $w$.$clientes$))
    
 \end{enumerate}
 Adentro de wolfieEstr.títulos:
 \begin{enumerate}
 	\item accionesDisponibles no puede ser mayor a maxAcciones.\\
    ($\forall nomTit:titulo$) (def?($nomTit$, $w$.$titulos$) $\Rightarrow$ cantidadMáximaAcciones($nomTit$, $w.titulos$) $\ge$\\ accionesDisponibles($nomTit$, $w.titulos$))
 \end{enumerate}
\BlankLine
Adentro de wolfieEstr.clientes:
 \begin{enumerate}
 	\item En cada infoCliente el totalAcciones tiene que ser igual a la suma de cantidadAcciones de todos los títulos definidos en infoCliente.títulos.\\
    ($\forall claveCliente:cliente$) (def?($claveCliente$, $w$.$clientes$) $\Rightarrow$ totalAcciones($claveCliente$, $w$.$clientes$) = \\ sumatoriaCantidadAcciones(dameTítulos($claveCliente$, $w$.$clientes$))
    \item En todas las entradas de infoTítuloCliente si promesas.venta.pendiente es verdadero entonces promesas.venta.cantidad tiene que ser menor o igual a el infoTítuloCliente.cantidadAcciones.\\
($\forall claveCliente:cliente$) (def?($claveCliente$, $w$.$clientes$) $\Rightarrow$ ($\forall nomTit:titulo$)\\ (def?($nomTit$, dameTítulos($claveCliente$, $w$.$clientes$)) $\Rightarrow$\\ cantidadPrometidasVenta(obtener($nomTit$, dameTítulos($claveCliente$, $w$.$clientes$)) $\leq$\\cantidadAcciones(obtener($nomTit$, dameTítulos($claveCliente$, $w$.$clientes$))
 \end{enumerate}
 Adentro de wolfieEstr.promesasDe cuando promesasDe.actualizado sea verdadero:
 \begin{enumerate}
 	\item no puede haber más de una promesa de compra sobre cada título\\
    wolfieEstr.promesasDe.actualizado $\Rightarrow$ ($\forall nomTit:string$)\\ (cantidadDeCompra(wolfieEstr.promesasDe.promesas, nomTit) = 1)
    \item no puede haber más de una promesa de venta sobre cada título\\
    wolfieEstr.promesasDe.actualizado $\Rightarrow$ ($\forall nomTit:string$)\\(cantidadDeVenta(wolfieEstr.promesasDe.promesas, nomTit) = 1)
 \end{enumerate}

 Entre wolfieEstr.promesasDe y wolfieEstr.clientes cuando wolfieEstr.promesasDe.actualizado sea verdadero:
 \begin{enumerate}
 	\item promesasDe.cliente pertenece a los clientes de wolfie.\\
    wolfieEstr.promesasDe.actualizado $\Rightarrow$ def?(wolfieEstr.clientes, promesasDe.cliente) 
    \item todas las promesas en promesasDe.promesas están en el correspondiente infoCliente y viceversa.\\
promesasDe.actualizado $\Rightarrow$\\ esPermutacion(promesasALista(dameTítulosCliente(wolfieEstr.clientes, promesasDe.cliente)), promesasDe.promesas)
 \end{enumerate}

  \Rep[wolfie][w]{
  ($\forall nomTit:titulo$) (def?($nomTit$, $w$.$titulos$) $\Rightarrow$ cantidadMáximaAcciones($nomTit$, $w.titulos$) $\ge$ accionesDisponibles($nomTit$, $w.titulos$))
  $\land$
  ($\forall claveCliente:cliente$) (def?($claveCliente$, $w$.$clientes$) $\Rightarrow$ totalAcciones($claveCliente$, $w$.$clientes$) = sumatoriaCantidadAcciones(dameTítulos($claveCliente$, $w$.$clientes$))
  	$\land$
  	($\forall nomTit:titulo$) (def?($nomTit$, dameTítulos($claveCliente$, $w$.$clientes$)) $\Rightarrow$ cantidadPrometidasVenta(obtener($nomTit$, dameTítulos($claveCliente$, $w$.$clientes$)) $\leq$ cantidadAcciones(obtener($nomTit$, dameTítulos($claveCliente$, $w$.$clientes$))  		        
        $\land$ 
        def?($nomTit$, $w$.$titulos$)
        )
    )
   $\land$
   ($\forall nomTit:string$) (def?($nomTit$, $w$.$titulos$) $\Rightarrow$ dameDisponibles($nomTit$, $w$.$titulos$) = dameMaxAcciones($nomTit$, $w$.$titulos$) - sumatoriaAccionesTítulo($t$, $w$.$clientes$))
   $\land$
   wolfieEstr.promesasDe.actualizado $\Rightarrow$ (($\forall nomTit:string$) (cantidadDeCompra(wolfieEstr.promesasDe.promesas, nomTit) = 1 $\land$ cantidadDeVenta(wolfieEstr.promesasDe.promesas, nomTit) = 1)) $\land$ def?(wolfieEstr.clientes, promesasDe.cliente) $\land$ esPermutacion(promesasALista(dameTítulosCliente(wolfieEstr.clientes, promesasDe.cliente)), promesasDe.promesas)
 
        
  }\mbox{}
  
  ~
  
  \tadOperacion{sumatoriaAccionesTítulo}{t/string,clientes/dict({cliente, infoCliente})}{nat}{}
  \tadAxioma{sumatoriaAccionesTítulo($t$,$clientes$)}{sumatoriaAccionesTítuloConj($t$, claves($clientes$), $clientes$)}
  
  ~
  
  \tadOperacion{sumatoriaAccionesTítuloConj}{t/string,claves/conj(string), clientes/dict({nat, infoCliente})}{nat}{}
  \tadAxioma{sumatoriaAccionesTítuloConj($t$, $claves$, $clientes$)}{\IF $\emptyset$?(claves) THEN 0 ELSE 
  dameCantAcciones($t$, dameTítulos(obtener(dameUno($claves$), $clientes$))) + sumatoriaAccionesTítuloConj($t$, sinUno($claves$), $clientes$) FI}
  
  ~
  
  \tadOperacion{dameCantAcciones}{nomTit/string,títulos/{dict(nat, infoTituloCliente)}}{nat}{}
  \tadAxioma{dameCantAcciones($nomTit$,$títulos$)}{\IF def?($nomTit$, $titulos$) THEN obtener($nomTit$, $titulos$).cantidadAcciones ELSE 0 FI}
  
  ~
  
  \tadOperacion{dameMaxAcciones}{nomTit/string,títulos/{dict(string, infoTitulo)}}{nat}{def?($nomTit$, $titulos$)}
  \tadAxioma{dameMaxAcciones($nomTit$,$títulos$)}{$\prod_1$(obtener($nomTit$, $titulos$))}

  ~      

  \tadOperacion{cantidadMáximaAcciones}{nomTit/string,títulos/{dict(string, infoTitulo)}}{nat}{def?($nomTit$, $titulos$)}
  \tadAxioma{cantidadMáximaAcciones($nomTit$,$títulos$)}{obtener($nomTit$, $titulos$).maxAcciones}

  ~
  
  \tadOperacion{accionesDisponibles}{nomTit/string, títulos/{dict(string, infoTitulo)}}{nat}{def?($nomTit$, $titulos$)}
  \tadAxioma{accionesDisponibles($nomTit$,$títulos$)}{obtener($nomTit$, $titulos$).accionesDisponibles}
  
  ~
  
  \tadOperacion{dameTítulos}{c/nat, clientes/dict({nat, infoCliente})}{dict}{def?($c$, $clientes$)}
  \tadAxioma{dameTítulos($c$, $clientes$)}{obtener($c$, $clientes$).titulos}

  ~
  
  \tadOperacion{totalAcciones}{c/nat, clientes/{dict(nat, infoCliente)}}{dict}{def?($c$, $clientes$)}
  \tadAxioma{totalAcciones($c$, $clientes$)}{obtener($c$, $clientes$).totalAcciones}
  
  ~

  \tadOperacion{sumatoriaCantidadAcciones}{títulos/{dict(string, infoTítuloCliente)}}{nat}{}
  \tadAxioma{sumatoriaCantidadAcciones($titulos$)}{sumatoriaPrimeraComponenteDiccionario(claves($titulos$), $titulos$)}
  
  ~

  \tadOperacion{sumatoriaPrimeraComponenteDiccionario}{c/conj(string), d/{dict(string, infoTítuloCliente)}}{nat}{}
  \tadAxioma{sumatoriaPrimeraComponenteDiccionario($c$, $d$)}{\IF $\emptyset ?$($c$) THEN 0 ELSE $\prod_1$(obtener(dameUno($c$), $d$)) + 
  	sumatoriaPrimeraComponenteDiccionario(sinUno($c$), $d$) FI}

  ~

  \tadOperacion{cantidadPrometidasVentas}{t/infoTituloCliente}{nat}{}
  \tadAxioma{cantidadPrometidasVentas($t$)}{t.venta.cantidad}
  
  ~	

  \tadOperacion{cantidadAcciones}{t/infoTituloCliente}{nat}{}
  \tadAxioma{cantidadAcciones($t$)}{t.cantidadAcciones}
	
  ~    	

  \tadOperacion{cantidadDeCompra}{promesas/secu(promesaTítulo), nomTit/string}{nat}{}
  \tadAxioma{cantidadDeCompra($promesas$, $nomTit$)}{\IF vacia?(promesas) THEN
  0 
  ELSE {(\IF prim(promesas).nomTit = nomTit $\land$ prim(promesas).tipo = compra THEN 1 ELSE 0 FI)} + cantidadDeCompra(fin($promesas$), $nomTit$) FI}
	
  ~
  
  \tadOperacion{cantidadDeVenta}{promesas/secu(promesaTítulo), nomTit/string}{nat}{}
  \tadAxioma{cantidadDeVenta($promesas$, $nomTit$)}{\IF vacia?(promesas) THEN
  0 
  ELSE {(\IF prim(promesas).nomTit = nomTit $\land$ prim(promesas).tipo = venta THEN 1 ELSE 0 FI)} + cantidadDeVenta(fin($promesas$), $nomTit$) FI}
	
  ~
  
  \tadOperacion{dameTítulosCliente}{clientes/dict(nat, infoCliente), c/cliente}{dictTítulos}{}
  \tadAxioma{dameTítulosCliente($clientes$, $c$)}{obtener($clientes$, $c$).títulos}
	
  ~
  
  \tadOperacion{promesasALista}{titulos/dict(string, infoTítuloCliente)}{secu(promesaTítulo)}{}
  \tadAxioma{promesasALista($titulos$)}{promesasAListaConClaves(claves($titulos$), $titulos$)}
	
  ~
  
  \tadOperacion{promesasAListaConClaves}{claves/conj(string), titulos/dict(string, infoTítuloCliente)}{secu(promesaTítulo)}{}
  \tadAxioma{promesasAListaConClaves($claves$, $titulos$)}{\IF $\emptyset$?(claves) THEN <> ELSE generarPromesasTítulo(dameUno($claves$), obtener($titulos$, dameUno($claves$)) \& promesasAListaConClaves(sinUno(claves), $titulos$) FI}

  ~
  
  \tadOperacion{generarPromesasTítulo}{nomTit/string, info/infoTítuloCliente}{secu(promesaTítulo)}{}
  \tadAxioma{generarPromesasTítulo($nomTit$, $info$)}{(
  \IF $info$.$promesas$.$compra$.$pendiente$ THEN generarPromesaTítulo(nomTit,compra, $info$.$promesas$.$compra$) ELSE <> FI) \& (\IF $info$.$promesas$.$venta$.$pendiente$ THEN generarPromesaTítulo(nomTit, venta, $info$.$promesas$.$venta$) ELSE <> FI)}

  ~
  
  \tadOperacion{generarPromesaTítulo}{nomTit/string, tipo/string, p/promesa}{promesaTítulo}{}
  \tadAxioma{generarPromesasTítulo($nomTit$, $tipo$, $p$)}{<$nomTit$, $tipo$, $p$.$umbral$, $p$.$cantidad$>}

  ~
\BlankLine
\textbf{Función de abstracción}\\
  % asumo que we es el wolfie de diseño y w es el de tads
  \AbsFc[wolfie]{wolfie}{clientes($w$) = claves(we.clientes) $\land$ títulos($w$) = claves($we$.$titulos$) $\land$ 
  ($\forall c:cliente, t:titulo$) ($c$ $\in$ clientes($w$) $\land$ $t$ $\in$ titulos($w$) $\Rightarrow$ 
  accionesPorCliente($c$, nombre($t$), $w$) = dameCantAcciones(nombre($t$), dameTítulos($c$, $we$.$clientes$))) $\land$
  ($\forall c:cliente$) ($c$ $\in$ clientes($w$) $\Rightarrow$
  ($\forall p:promesa$) ($p \in$ promesasDe($c$, $w$) $\Leftrightarrow$ ($\exists$ $pEstr:promesaTítulo$ / $pEstr$ $\in$ promesasAConj($c$, $we$.$clientes$)
  $\land$ $tp$.$tipo$ = tipo($p$) $\land$ $tp$.$umbral$ = limite($p$)
  $\land$ $tp$.$cantidad$ = cantidad($p$) $\land$ $tp$.$nomTit$ = titulo($p$))))
  }
  
  ~
  
  \tadOperacion{promesasAConj}{c/cliente, clientes/{dict(nat, infoCliente)}}{conj(tPromesa)}{}
  \tadAxioma{promesasAConj($c$, $clientes$)}{damePromesas(obtener($clientes$, $c$).$titulos$)}
  
  ~
  
  \tadOperacion{damePromesas}{titulos/dict(string, infoTítuloCliente)}{conj(promesaTítulo)}{}
  \tadAxioma{damePromesas($titulos$)}{promesasAConjConClaves(claves($titulos$), $titulos$)}
	
  ~
  
  \tadOperacion{promesasAConjConClaves}{claves/conj(string), titulos/dict(string, infoTítuloCliente)}{secu(promesaTítulo)}{}
  \tadAxioma{promesasAConjConClaves($claves$, $titulos$)}{\IF $\emptyset$?(claves) THEN $\emptyset$ ELSE generarPromesasTítulo(dameUno($claves$), obtener($titulos$, dameUno($claves$)) $\cup$ promesasAConjConClaves(sinUno(claves), $titulos$) FI}

  ~
  
  \tadOperacion{generarPromesasTítulo}{nomTit/string, info/infoTítuloCliente}{secu(promesaTítulo)}{}
  \tadAxioma{generarPromesasTítulo($nomTit$, $info$)}{(
  \IF $info$.$promesas$.$compra$.$pendiente$ THEN generarPromesaTítulo(nomTit,compra, $info$.$promesas$.$compra$) ELSE $\emptyset$ FI) $\cup$ (\IF $info$.$promesas$.$venta$.$pendiente$ THEN generarPromesaTítulo(nomTit, venta, $info$.$promesas$.$venta$) ELSE $\emptyset$ FI)}

  ~
  
  \tadOperacion{generarPromesaTítulo}{nomTit/string, tipo/string, p/promesa}{promesaTítulo}{}
  \tadAxioma{generarPromesasTítulo($nomTit$, $tipo$, $p$)}{<$nomTit$, $tipo$, $p$.$umbral$, $p$.$cantidad$>}

  ~
  
  \tadOperacion{dameCantAcciones}{nomTit/string,títulos/{dict(string, infoTituloCliente)}}{nat}{}
  \tadAxioma{dameCantAcciones($nomTit$,$titulos$)}{\IF def?($nomTit$, $titulos$) THEN $\prod_1$(obtener($nomTit$, $titulos$)) ELSE 0 FI}
  
  ~
  
  \tadOperacion{dameTítulos}{c/nat, clientes/{dict(nat, infoCliente)}}{dict(string, infoTituloCliente)}{def?($c$, $clientes$)}
  \tadAxioma{dameTítulos($c$, $clientes$)}{$\prod_1$(obtener($c$, $clientes$))}
  
  ~
  
  \begin{Algoritmos}
\\
\begin{algorithm}[H]
\textit{i}InaugurarWolfie(\In{clientes}{conj(cliente)})\\
\BlankLine
diccTítulos w.títulos $\leftarrow$ NuevoDiccionario()\tcp*{O( 1 )}
diccClientes w.clientes $\leftarrow$ Vacío()\tcp*{O( 1 )}
itConj itClientes = crearIt(clientes)\tcp*{O( 1 )}
\While {haySiguiente(itClientes)}{\tcp*{O(\#(clientes))}
	diccTítulos títulos $\leftarrow$ NuevoDiccionario()\tcp*{O( 1 )}
	Definir(w.clientes, siguiente(itClientes), <títulos, 0>)\tcp*{O( 1 )}
    avanzar(itClientes)\tcp*{O( 1 )}
}
\end{algorithm}

\begin{algorithm}[H]
\textit{i}AgregarTítulo(\Inout{w}{wolfie}, \In{nomTit}{string}, \In{maxAcciones}{nat}, \In{cot}{nat})\\
\BlankLine
nat accionesDisponibles $\leftarrow$ maxAcciones\\
bool enAlza $\leftarrow$ true\\
nat rachaActual $\leftarrow$ 0\\
nat rachaMaxima $\leftarrow$ 0\\
nat fluctuaciones $\leftarrow$ 0\\
tupla infoTítulo $\leftarrow$ <maxAcciones, accionesDisponibles, cot, enAlza, rachaActual, rachaMaxima, fluctuaciones>\\
Definir(nomTit, infoTítulo, w.títulos)\tcp*{O( |nomTit| )}
\end{algorithm}

\begin{algorithm}[H]
\textit{i}ActualizarCotización(\Inout{w}{wolfie}, \In{nomTit}{string}, \In{cot}{nat})\\
\BlankLine
w.promesasDe.actualizado $\leftarrow$ false\\
\BlankLine
tupla infoTítulo $\leftarrow$ Obtener(w.títulos, nomTit)\tcp*{O( |nomTit| )}
\If{infoTítulo.cotización > cot}{
	{\If{infoTitulo.enAlza}{infoTitulo.fluctuaciones $\leftarrow$ infoTitulo.fluctuaciones + 1}}
	infoTítulo.enAlza $\leftarrow$ false
	infoTítulo.rachaActual $\leftarrow$ 0\\
}
\Else{
	{\If{!infoTitulo.enAlza}{infoTitulo.fluctuaciones $\leftarrow$ infoTitulo.fluctuaciones + 1}}
	infoTítulo.enAlza $\leftarrow$ true\\
	infoTítulo.rachaActual $\leftarrow$ infoTítulo.rachaActual + 1\\
	{\If{infoTítulo.rachaActual > infoTitulo.rachaMaxima}{infoTitulo.rachaMaxima $\leftarrow$ infoTitulo.rachaActual}}	
}
infoTítulo.cotización $\leftarrow$ cot\\
\BlankLine
// ejecutamos todas las promesas de venta\\
itDiccClientes itClientes $\leftarrow$ crearIt(w.clientes)\\
\While{haySiguiente(itClientes)}{\tcp*{O( \#clientes )}
	tupla infoCliente $\leftarrow$ siguienteSignificado(itClientes)\\
    \If {definido(nomTit, infoCliente.títulos)}{\tcp*{O( |nomTit| )}
    	tupla títutloActual $\leftarrow$ obtener(infoCliente.títulos, nomTit)\tcp*{O( |nomTit| )}
    	nat accionesVendidas $\leftarrow$ ejecutarVenta(títuloActual.promesas, cot)\tcp*{O( 1 )}
        \If{accionesVendidas > 0}{
        	títuloActual.cantidadAcciones $\leftarrow$ títuloActual.cantidadAcciones - accionesVendidas\\
            infoCliente.totalAcciones $\leftarrow$ infoCliente.totalAcciones - accionesVendidas\\
            infoTítulo.accionesDisponibles $\leftarrow$ infoTítulo.accionesDisponibles + accionesVendidas\\
        }
    }
}
\BlankLine
//generamos un arreglo de tuplas <cliente, totalAcciones> ordenado por la segunda\\ 
//componente de las tuplas\\
itDiccClientes itClientes $\leftarrow$ crearIt(w.clientes)\tcp*{O( 1 )}
nat cantidadClientes $\leftarrow$ \#claves(w.clientes)\\
arr clientesPorAcciones $\leftarrow$ crearArreglo(cantidadClientes)\tcp*{O( \#clientes )}
nat i $\leftarrow$ 0\\
\While{haySiguiente(itClientes)}{\tcp*{O( \#clientes )}
	tupla clienteTotalAcciones $\leftarrow$ <siguienteClave(itClientes), siguienteSignificado(itClientes).totalAcciones>\\
}
clientesPorAcciones $\leftarrow$ mergeSort(clientesPorAcciones)\tcp*{O( 1 )}
\BlankLine
// ejecutamos todas las promesas de compra\\
\For{nat i $\leftarrow$ 0 \KwTo (cantidadClientes-1)}{
	tupla infoCliente $\leftarrow$ obtener(w.clientes, clientesPorAcciones[i])\\
    \If{definido(nomTit, infoCliente.títulos)}{
    	tupla títuloActual $\leftarrow$ obtener(infoClientes.títulos, nomTit)\\
        nat accionesCompradas $\leftarrow$ ejecutarCompra(títuloActual.promesas, cot)\\
        \If{accionesCompradas > 0}{
        	títuloActual.cantidadAcciones $\leftarrow$ títuloActual.cantidadAcciones + accionesCompradas\\
            infoCliente.totalAcciones $\leftarrow$ infoCliente.totalAcciones + accionesCompradas\\
            infoTítulo.accionesDisponibles $\leftarrow$ infoTítulo.accionesDisponibles - accionesCompradas\\
        }
    }
    
}
\end{algorithm}

\begin{algorithm}[H]
\textit{i}EjecutarVenta(\Inout{promesas}{promesas},\In{cot}{nat}) $\longrightarrow$ res:nat \\
\If{promesas.venta.pendiente $\yluego$ promesas.venta.umbral < cot}{
	promesas.venta.pendiente $\leftarrow$ false\\
    res $\leftarrow$ promesas.venta.cantidad\\
\Else{
	res $\leftarrow$ 0\\	
}}
\BlankLine
\end{algorithm}

\begin{algorithm}[H]
\textit{i}EjecutarCompra(\Inout{promesas}{promesas},\In{cot}{nat}) $\longrightarrow$ res:nat \\
\If{promesas.compra.pendiente $\yluego$ promesas.compra.umbral < cot}{
	promesas.compra.pendiente $\leftarrow$ false\\
    res $\leftarrow$ promesas.compra.cantidad\\
\Else{
	res $\leftarrow$ 0\\	
}}
\BlankLine
\end{algorithm}


\begin{algorithm}[H]
\textit{i}AgregarPromesa(\Inout{w}{wolfie},\In{c}{cliente}, \In{nomTit}{string}, \In{tipo}{string}, \In{umbral}{nat}, \In{cantidad}{nat})\\
\BlankLine
\If{c = w.promesasDe.cliente}{w.promesasDe.actualizado $\leftarrow$ false}
\BlankLine
tupla infoCliente $\leftarrow$ obtener(c, w.clientes)\\
\If{definido(infoCliente.títulos, nomTit)}{
	tupla infoTítuloCliente $\leftarrow$ obtener(infoCliente.títulos, nomTit)\\
    \If{tipo = venta}{
    	infoTítuloCliente.promesas.venta $\leftarrow$ <true, umbral, cantidad>\\
    }
    \If{tipo = compra}{
    	infoTítuloCliente.promesas.compra $\leftarrow$ <true, umbral, cantidad>\\
    }    
}
\Else{
	tupla infoTítuloCliente\\
    infoTítuloCliente.cantidadAcciones $\leftarrow$ 0\\
    \If{tipo = venta}{
    	infoTítuloCliente.promesas.compra $\leftarrow$ <false, 0, 0>\\
    	infoTítuloCliente.promesas.venta $\leftarrow$ <true, umbral, cantidad>\\
    }
    \If{tipo = compra}{
    	infoTítuloCliente.promesas.venta $\leftarrow$ <false, 0, 0>\\
    	infoTítuloCliente.promesas.compra $\leftarrow$ <true, umbral, cantidad>\\
    }        
	definir(infoCliente.títulos, nomTit, infoTítuloCliente)\\
}
\end{algorithm}

\begin{algorithm}[H]
\textit{i}clientes() $\longrightarrow$ res: itDiccClientes(nat)\\
\BlankLine
res $\leftarrow$ crearItDiccOrd(w.clientes)
\end{algorithm}

\begin{algorithm}[H]
\textit{i}títulos() $\longrightarrow$ res: itDiccTítulos(nat)\\
\BlankLine
res $\leftarrow$ crearItDiccTítulos(w.títulos)
\end{algorithm}

\begin{algorithm}[H]
\textit{i}PromesasDe(\Inout{w}{wolfie}, \In{c}{cliente}) $\longrightarrow$ res: itLst(promesasTítulo)\\
\BlankLine
\If{c = w.promesasDe.cliente $\land$ w.promesasDe.actualizado}{
	res $\leftarrow$ crearIt(w.promesasDe.promesas)
}
\BlankLine
tupla infoCliente $\leftarrow$ obtener(w.clientes, c)\\
itDiccTítulos itTítulos $\leftarrow$ crearItDiccTítulos(infoCliente.títulos)\\
lista promesas $\leftarrow$ vacia()\\
\While{haySiguiente(itTítulos)}{
	tupla infoTítulo $\leftarrow$ siguienteSignificado(itTítulos)\\
    \If{infoTítulo.promesas.venta.pendiente}{
    	agregarAdelante(promesas, <siguienteClave(itTítulos), venta, infoTítulo.promesas.venta.umbral, infoTítulo.promesas.venta.cantidad>\\
    }
    \If{infoTítulo.promesas.compra.pendiente}{
    	agregarAdelante(promesas, <siguienteClave(itTítulos), compra, infoTítulo.promesas.compra.umbral, infoTítulo.promesas.compra.cantidad>\\
    }
}
\BlankLine
w.promesasDe.actualizado $\leftarrow$ true\\
w.promesasDe.cliente $\leftarrow$ c\\
w.promesasDe.promesas $\leftarrow$ promesas\\
\BlankLine
res $\leftarrow$ crearIt(promesas)\\
\end{algorithm}

\begin{algorithm}[H]
\textit{i}AccionesPorCliente(\In{w}{wolfie}, \In{nomTit}{string}, \In{c}{cliente}) $\longrightarrow$ res: nat\\
\BlankLine
tupla infoCliente $\leftarrow$ obtener(w.clientes, c)\\
tupla infoTítulo $\leftarrow$ obtener(infoCliente.títulos, nomTit)\\
res $\leftarrow$ infoTítulo.cantidadAcciones\\
\end{algorithm}

\begin{algorithm}[H]
\textit{i}EnAlza(\In{w}{wolfie}, \In{nomTit}{string}) $\longrightarrow$ res: bool\\
\BlankLine
tupla infoTítulo $\leftarrow$ obtener(w.títulos, nomTit)\\
res $\leftarrow$ infoTítulo.enAlza)\\
\end{algorithm}

\begin{algorithm}[H]
\textit{i}maximaRacha(\In{w}{wolfie}, \In{nomTit}{string}) $\longrightarrow$ res: nat\\
\BlankLine
tupla infoTítulo $\leftarrow$ obtener(w.títulos, nomTit)\\
res $\leftarrow$ infoTítulo.maximaRacha\\
\end{algorithm}

\begin{algorithm}[H]
\textit{i}tituloMasVolatil(\In{w}{wolfie}, \In{nomTit}{string}) $\longrightarrow$ res: nomTit\\
\BlankLine
itTitulos $\leftarrow$ crearIt(w.titulos)\\
nomTitMaximo $\leftarrow$ siguiente(itTitulos)\\
maxFluctuacion $\leftarrow$ Obtener(nomTitMaximo, w.titulos).fluctuaciones\\
avanzar(itTitulos)\\
\While{haySiguiente?(itTitulos)}{
	flucActual $\leftarrow$ Obtener(siguiente(itTitulos), w.titulos).fluctuaciones\\
	\If{ flucActual > maxFluctuacion}{
		maxFluctuacion $\leftarrow$ flucActual\\
		nomTitMaximo $\leftarrow$ siguiente(itTitulos)\\
	}
}
res $\leftarrow$ nomTitMaximo\\
\end{algorithm}



\end{Algoritmos}
\end{Representacion}

\section{Módulo Diccionario Títulos }
%\section{Módulo Diccionario Trie($\alpha$)}
%Estado: En proceso
%---------------------------------------------------------------
%El Tad Títulos se representa con un diccionario de tuplas
%El Diccionario se representa con Trie.
%Clave: nombre del título.
%Significado: Tupla < #maxAcciones, accionesDisponibles, cotización, enAlza >
%#maxAcciones, accionesDisponibles, cotización son Nat.
%enAlza es bool.
%---------------------------------------------------------------
\SetKw{Orden}{Orden Complejidad:}
\begin{Interfaz}
  
  \textbf{parámetros formales}\hangindent=2\parindent\\
  \parbox{1.7cm}{\textbf{géneros}} $\alpha$\\
  \parbox[t]{1.7cm}{\textbf{función}}\parbox[t]{\textwidth-2\parindent-1.7cm}{%
    \InterfazFuncion{Copiar}{\In{a}{$\alpha$}}{$\alpha$}
    {$res \igobs a$}
    [$\Theta(copy(a))$]
    [función de copia de $\alpha$'s]
  }

  \textbf{se explica con}: \tadNombre{Diccionario(String, $\alpha$)}.

  \textbf{géneros}: \TipoVariable{diccTítulo($String$,$\alpha$)}, \TipoVariable{itDicc$(\alpha)$}.

  \Title{Operaciones básicas de diccionario títulos}

  \InterfazFuncion{NuevoDiccionario}{}{diccTítulo($String$,$\alpha$)}%
  {$res \igobs$ vacio()}%
  [$\Theta(1)$]
  [Crea un nuevo diccionario vacío.]
  
  \InterfazFuncion{Definir}{\In{c}{string}, \In{s}{$\alpha$}, \Inout{d}{diccTítulo($String$,$\alpha$)}}{}
  [$d$ $\igobs$ $d_{0}$]
  {$d$ $\igobs$ definir($d_{0}$, $c$, $s$)}
  [$\Theta(|c| + copy(s))$]
  [Define la clave $c$ con el significado $s$ en el diccionario $d$.]
  [Se agrega por copia el significado $s$] 

  \InterfazFuncion{Obtener}{\In{c}{string}, \In{d}{diccTítulo($String$,$\alpha$)}}{$\alpha$}
  [def?($c$, $d$)]
  {$res$ $\igobs$ obtener($c$, $d$)}
  [$\Theta(|c|)$]
  [Devuelve el significado de la clave $c$ contenida en el diccionario $d$.]
  []
  
  \InterfazFuncion{Definido?}{\In{c}{string}, \In{d}{diccTítulo($String$,$\alpha$)}}{$bool$}
  {$res$ $\igobs$ def?($c$, $d$)}
  [$\Theta(|c|)$]
  [Chequea si está definida la clave $c$ en el diccionario $d$.]
  []

  \InterfazFuncion{\#Claves}{\In{d}{diccTítulo($String$,$\alpha$)}}{nat}
  {$res$ $\igobs$ \#(claves($d$))}
  [$\Theta(1)$]
  [Devuelve la cantidad de claves del diccionario.]
  
%----------------------------------------------------------
% Operaciones del Iterador
% Idem a las de diccionario lineal del apunte de modulos básicos
% Iteramos como si fuera un diccionario lineal O(1)
% Buscamos y obtenemos utilizando la estructura trie O(|c|)

  \Title{Operaciones del iterador}
  \InterfazFuncion{CrearIt}{\In{d}{diccTítulo($String$,$\alpha$)}}{itDicc($String$,$\alpha$)}
  {$res$ $\igobs$ < crearItUni(<>,$d$.claves), crearItUni(<>,$d$.alfas)>}
  [$\Theta(1)$]
  [Crea un iterador unidireccional del diccionario. Se pueden recorrer los elementos aplicando iterativamente Siguiente]
  % bidireccional del diccionario, de forma tal que $\NombreFuncion{HayAnterior}$ evalúe a $\ensuremath{\mathit{false}}$ (ie. que se pueda recorrer los elementos aplicando iterativamente $\NombreFuncion{Siguiente}$).]
  
  \InterfazFuncion{HaySiguiente?}{\In{it}{itDicc($String$,$\alpha$)}}{bool}
  {$res$ $\igobs$ haySiguiente?($it$)}
  [$\Theta(1)$]
  [Devuelve true si y sólo si en el iterador todavía quedan elementos para avanzar]
    
  %\InterfazFuncion{Actual}{\In{it}{itDicc$(\alpha)$}}{String}
  %[HaySiguiente?($it$)]
  %{alias($res$ $\igobs$ Actual($it$)}
  %%{alias($res$ $\igobs$ Siguiente($it$))}
  %[$\Theta(1)$]
  %[Devuelve el elemento actual del iterador (tupla: clave - puntero a significado)]
  %%[La clave no es modificable]

  %\InterfazFuncion{ActualClave}{\In{it}{itDicc$(\alpha)$}}{String}
  %[HaySiguiente?($it$)]
  %{alias($res$ $\igobs$ $\prod_1$(Actual($it$))}
  %%{alias($res$ $\igobs$ Siguiente($it$))}
  %[$\Theta(1)$]
  %[Devuelve la clave actual del iterador]
  %%[La clave no es modificable]

  %\InterfazFuncion{ActualSignificado}{\In{it}{itDicc$(\alpha)$}}{$\alpha$}
  %[HaySiguiente?($it$)]
  %{alias($res$ $\igobs$ $\prod_2$(Actual($it$))}
  %%{alias($res$ $\igobs$ Siguiente($it$))}
  %[$\Theta(1)$]
  %[Devuelve el significado actual del iterador]

  \InterfazFuncion{Siguiente}{\In{it}{itDicc($String$,$\alpha$)}}{tupla($String$, $\alpha$)}
  [HaySiguiente?($it$)]
  {alias($res$ $\igobs$ Siguiente($it$))}
  [$\Theta(1)$]
  [Devuelve el elemento siguiente del iterador]
  [$res$.significado es un puntero al objeto $\alpha$ y es modificable si y sólo si $it$ es modificable. En cambio, $res$.clave no es modificable] 

  \InterfazFuncion{SiguienteClave}{\In{it}{itDicc($String$,$\alpha$)}}{String}
  [HaySiguiente?($it$)]
  {alias($res$ $\igobs$ Siguiente($it$).clave)}
  [$\Theta(1)$]
  [Devuelve la clave del elemento siguiente del iterador]
  [$res$ no es modificable] %PREGUNTAR!!!

  \InterfazFuncion{SiguienteSignificado}{\In{it}{itDicc($String$,$\alpha$)}}{$\alpha$}
  [HaySiguiente?($it$)]
  {alias($res$ $\igobs$ Siguiente($it$))}
  [$\Theta(1)$]
  [Devuelve el significado del elemento siguiente del iterador]
  [$res$ es modificable si y sólo si $it$ es modificable]
  
  \InterfazFuncion{Avanzar}{\Inout{it}{itDicc($String$,$\alpha$)}}{}
  [$it = it_0$ $\land$ HaySiguiente?($it$)]
  {$it$ $\igobs$ Avanzar($it_0$)}
  [$\Theta(1)$]
  [Avanza a la posición siguiente del iterador]
  
  
\end{Interfaz}

\begin{Representacion}\\
\tab\textbf{Representación del diccionario:}
  \begin{Estructura}{diccTítulo($String$, $\alpha$)}[estr\_diccTítulo]
    \begin{Tupla}[estr\_diccTítulo]
      \tupItem{raíz}{estr\_nodo}
      \tupItem{claves}{lista(String)}
    \end{Tupla}
      
    \begin{Tupla}[estr\_nodo]
      \tupItem{significado}{puntero $(\alpha)$}
      \tupItem{hijos}{array[256] de puntero(estr\_nodo)}%
    \end{Tupla}
  \end{Estructura}  

 \textbf{Invariante de representación:}
 \begin{enumerate}
 	\item Dos nodos no pueden compartir un hijo
    \item Sin ciclos en el árbol
    \item Las hojas del árbol no pueden tener significado nulo
    \item La cantidad de claves ingresadas en e.claves debe ser igual a la cantidad de significados válidos (distintos de NULL) del árbol Trie
    \item Las claves contenidas en $e$.$claves$ deben estar definidas en el arbol Trie
 \end{enumerate}

\end{Representacion}

\Rep[estr\_diccTítulo][e]{ \\
    SinCiclos($e$.raíz, $\emptyset$) $\yluego$ \hfill 1. \\
    NoCompartenHijos($e$.raíz) $\yluego$ \hfill 1. \\
    SignificadosHojasNotNull($e$.raíz) $\land$ \hfill 2. \\    
    \TipoVariable {Longitud($e$.claves)} $\igobs$ CantSignificados($e$.raíz) $\land$ \hfill 3. \\
    ($\forall clave:String$) ((está?($clave$, $e$.claves)) $\Leftrightarrow$ definido?($clave$,$e$.raíz)  )\hfill 4.\\
  }
 ~

\tadOperacion{SignificadoHojaNotNull}{estr\_nodo/e}{bool}{}
\tadAxioma{SignificadoHojaNotNull($e$)}
{\IF Hoja?($e$, 0) THEN 
	$\neg$(significado($e$)$\igobs$ NULL)
  ELSE
  	RecorrerHijos($e$,0)
  FI}
\tadOperacion{Hoja?}{estr\_nodo/e, nat}{bool}{}
\tadAxioma{Hoja?($e$,$n$)}
{($e$.hijos[$n$] $\igobs$ NULL) $\land$
\IF ($n$ < 256) THEN
	Hoja?($e$,$n$+1)
ELSE
	true
FI}

~

\tadOperacion{RecorrerHijos}{estr\_nodo/e, nat}{bool}{SinCiclos($e$,0)}
\tadAxioma{RecorrerHijos($e$,$n$)}
{(\IF ($\neg$($e$.hijos[$n$] $\igobs$ NULL)) THEN
	SignificadoHojaNotNull(*(e.hijos[$n$]))
  ELSE
   true
  FI) $\land$
	({\IF ($n$ < 256) THEN 
		RecorrerHijos($e$,$n$+1)
	ELSE
		true
	FI)}
FI}

\tadOperacion{CantSignificados}{estr\_nodo/$e$}{nat}{}
\tadAxioma{CantSignificados($e$)}
{(\IF ($e$.significado $\igobs$ NULL) THEN
	0
ELSE
	1
FI) + SigHijos(e,0)}

\tadOperacion{SigHijos}{estr\_nodo/$e$, nat}{nat}{}
\tadAxioma{SigHijos($e$,$n$)}
{(\IF ($e$.hijos[$n$] $\igobs$ NULL) THEN
	0
ELSE
	CantSignificados(*($e$.hijos[$n$]))
FI) + (\IF ($n$ < 256) THEN
	SigHijos($e$,$n$+1)
ELSE
	0
FI)}

%Abstraccion DICTIONARIO
\textbf{Función de abstracción:}
\Abs[estr\_diccTítulo]{dicc(string, $\alpha$)}[e]{d}{
    \#(claves($d$)) $\igobs$ \TipoVariable {Longitud($e$.clave)} $\yluego$ \hfill 1. \\
    ($\forall c$: string)(def?($c$, $d$) $\impluego$  \hfill 2. \\
       $\tab$ (definido?($c$, $e$.raíz) $\yluego$  \\
       $\tab \tab$ obtener($c$, $d$) $\igobs$ *(ObtDeEstruc($c$, $e$.raíz)) )
    )
}

\tadOperacion{definido?}{string/e, estr\_nodo/$e$}{$bool$}{}
\tadAxioma{definido?($c$,$n$)}
{\IF vacía?($c$) THEN
	$\neg$(n.significado = NULL)
ELSE
	{\IF n.hijos[ORD(prim(c)] = NULL THEN false 
    ELSE 
    	definido?(fin($c$),$n$.hijos[ORD(prim($c$))]    
    FI}
FI}
 
\tadOperacion{ObtDeEstruc}{string/$ e$, estr\_nodo/$e$}{$\alpha$}{}
\tadAxioma{ObtDeEstruc($c$,$n$)}
{\IF vacía?($c$) THEN
	n.significado
ELSE		
   	ObtDeEstruc(fin($c$),$n$.hijos[ORD(prim($c$))]    
FI}

%REPresentacion iterador
 \textbf{Representación del iterador:}\\ \tab
 El iterador del diccionario es simplemente un iterador a la lista de claves.\\\tab Lo único que hay que pedir es que satisfaga el Rep de esta lista.\\\tab Por implementación, alcanza con que sea unidireccional. 
 	\begin{Estructura}{itDiccTítulos($String$,$\alpha$)}[itDic]
    \begin{Tupla}[itDic]
		\tupItem{claves}{itLista($String$)}
    \end{Tupla}
 	\end{Estructura}

\Rep[itDic][it]{ \\
	Rep(it.claves)
  }
Abs : itDic $it$ $\longrightarrow$ itUni($String$) \hfill\{Rep($it$)\}\\ \hspace*{4mm}

Abs($it$) $\equiv$ CrearItUni(Siguientes(it.claves)))

%%%%%%%%%%%%%%%%%%%%%%%%%%%%%%%%%%%%%%%%%%%%%%%%%%%%%%%%%%%%%%%%%%%%%%%%%%%%%%%%%%%%%%%%%%%%%%%%%%%%%%%%%%%%%%%%%%%%%%%%%%%%%%%%%%%%%%%%%%%%%%%%%%
%ALGORITMOS
\begin{Algoritmos}
 \incmargin{1em}
    %\linesnumbered
   %\restylealgo{boxed}
    \dontprintsemicolon
\\
%NuevoDiccionario
\tab\textit{i}NuevoDiccionario() $\longrightarrow$ res: bool\\
\hspace*{9mm}\Orden{O(1)}\\
\begin{algorithm}[H]
%\BlankLine
	res $\leftarrow$ <raíz: iNodoNuevo(), Claves : Vacía(), Alfas : Vacía() > \tcp*{O(1)}
\end{algorithm}

%Nodo Nuevo
\textit{i}NodoNuevo() $\longrightarrow$ res: estr\_nodo\\
\hspace*{9mm}\Orden{O(1)}\\
\begin{algorithm}[H]
%\BlankLine
	res $\leftarrow$ <significado: NULL, hijos:CrearArreglo() > \tcp*{O(1)}
	\For{var $i : nat\leftarrow 0$ \KwTo $255$\tcp*{O(256)}}{
		res.hijos[i] $\leftarrow$ NULL 
	}
\end{algorithm}

%Definir
\textit{i}Definir(\In{c}{string}, \In{s}{$\alpha$}, \Inout{d}{estr\_diccTítulo}) $\longrightarrow$ res: estr\_dicTrie\\
\hspace*{9mm}\Orden{O(|c|)}\\
\begin{algorithm}[H]
	var $actual$ : \TipoVariable{puntero($estr\_nodo$)} $\leftarrow$ \&($d$.raíz) \tcp*{O( 1 )}
	\For{var $i : nat\leftarrow 0$ \KwTo (Longitud($c$) - 1)\tcp*{O(|c|)}}{ 
		\If{actual $\rightarrow$ hijos[ORD(c[i])] = NULL}{
				AgregarAtrás(d.$Claves$, c) \tcp*{O(|c|)}
			actual $\rightarrow$ hijos[ORD(c[i])] $\leftarrow$ $\&$(iNodoNuevo())
		}
		actual $\leftarrow$ (actual $\rightarrow$ hijos[ORD(c[i])])
	}
	actual $\rightarrow$ significado $\leftarrow \&$(Copiar($s$))\tcp*{O( 1 )} 
\end{algorithm}

%Obtener
\textit{i}Obtener(\In{c}{string}, \In{d}{estr\_diccTítulo}) $\longrightarrow$ res:$\alpha$ \\
\hspace*{9mm}\Orden{O(|c|)}\\
\begin{algorithm}[H]
%\BlankLine
	var $actual$ : \TipoVariable{puntero($estr\_nodo$)} $\leftarrow$ \&($d$.raíz)\tcp*{O( 1 )}
	\For{var $i : nat\leftarrow 0$ \KwTo (Longitud($c$) - 1)\tcp*{O(|c|)}}{
		actual $\leftarrow$ (actual $\rightarrow$ hijos[ORD(c[i])])
	}
	res $\leftarrow$ * (actual $\rightarrow$ significado)\tcp*{O( 1 )} 
\end{algorithm}

%Definido
% Ejemplo: Si esta definido CASA
% Casos: 
%	1 - puede estar definido CASAS pero no CASA -> Significado en CASA = NULL al salir del while
%	2 - puede estar definido CAS o algo menor a CASA:  actual -> NULL al salir del while
%	3 - pueden estar definido CASA y CASAS: sale del while por i < longitud(c), no podemos chequear solo que actual != NULL 
%	4 - No puedo pedir actual->significado sin antes verificar que actual no apunte a NULL 

\textit{i}Definido?(\In{c}{string}, \In{d}{estr\_diccTítulo})$\longrightarrow$ res: bool \\
\hspace*{9mm}\Orden{O(|c|)}\\
\begin{algorithm}[H]
%\BlankLine
	var $actual$ : \TipoVariable{puntero($estr\_nodo$)} $\leftarrow$ \&($d$.raíz) \tcp*{O( 1 )}
	var i : \TipoVariable{Nat} $\leftarrow$ $0$ \tcp*{O( 1 )}
	\While{ $actual$ != NULL $\&\&$  ($i$ < Longitud($c$))\tcp*{O(|c|)}}{
		$i$ $\leftarrow$ $i$ + 1 \\
		actual $\leftarrow$ (actual $\rightarrow$ hijos[ORD(c[i])])
	}
	\eIf{($actual$ != NULL)}{
		\eIf{($actual$ $\rightarrow$ significado) != NULL}{
			res $\leftarrow$ $true$\tcp*{O( 1 )}
		}{
			res $\leftarrow$ $false$\tcp*{O( 1 )}
		}
	}{
		res $\leftarrow$ $false$\tcp*{O( 1 )}
	}

\end{algorithm}

%Cantidad de Claves
\textit{i}$\#$Claves(\In{d}{estr\_diccTítulo})$\longrightarrow$ res: nat\\
\hspace*{9mm}\Orden{O(1)}\\
\begin{algorithm}[H]
%\BlankLine
	res $\leftarrow$ Longitud($d$.Claves)\tcp*{O( 1 )}
\end{algorithm}

\textbf {Algoritmos del iterador}

\textit{i}CrearIT(in d: diccTítulo($String$,$\alpha$)) $\longrightarrow$ res: ItDicc($String$,$\alpha$)\\
\hspace*{9mm}\Orden{O(1)}\\
\begin{algorithm}[H]
\BlankLine
res $\leftarrow$ claves:CrearIt($d$.claves)
\end{algorithm}

\textit{i}HaySiguiente?(in it: ItDicc($String$,$\alpha$)) $\longrightarrow$ res: bool\\
\hspace*{9mm}\Orden{O(1)}\\
\begin{algorithm}[H]
\BlankLine
res $\leftarrow$ HaySiguiente(it.claves)
\end{algorithm}

\textit{i}Siguiente(in it: ItDicc($String$,$\alpha$)) $\longrightarrow$ res: $String$\\
\hspace*{9mm}\Orden{O(1)}\\
\begin{algorithm}[H]
\BlankLine
res $\leftarrow$ Siguiente(it.claves)
\end{algorithm}

\textit{i}Avanzar(in it: ItDicc($String$,$\alpha$)) $\longrightarrow$  res: ItDicc($String$)\\
\hspace*{9mm}\Orden{O(1)}\\
\begin{algorithm}[H]
\BlankLine
res $\leftarrow$ Avanzar(it.claves)
\end{algorithm}

\end{Algoritmos}

\section{Módulo Diccionario Clientes}


\begin{Interfaz}
  
  \textbf{parámetros formales}\hangindent=2\parindent\\
  \parbox{1.7cm}{\textbf{géneros}} $\alpha$\\
  \parbox[t]{1.7cm}{\textbf{función}}\parbox[t]{\textwidth-2\parindent-1.7cm}{%
    \InterfazFuncion{Copiar}{\In{a}{$\alpha$}}{$\alpha$}
    {$res \igobs a$}
    [$\Theta(copy(a))$]
    [función de copia de $\alpha$'s]
  }

  \textbf{se explica con}: \tadNombre{Diccionario($\alpha$,$\sigma$)}, \tadNombre{Iterador Unidireccional(Tupla($\alpha$,$\sigma$))}.

  \textbf{géneros}: \TipoVariable{dictClientes(nat, infoCliente)}, \TipoVariable{itDictClientes(Tupla($\alpha$,$\sigma$))}.

  \textbf{Operaciones básicas de diccionario ordenado}

  \InterfazFuncion{Vacío}{\In{n}{nat}}{dictClientes(nat, infoCliente)}%
  {$res \igobs vacio$}%
  [$O(n)$]
  [genera un diccionario vacío.]

  \InterfazFuncion{Definir}{\Inout{d}{dict(nat, infoCliente)}, \In{c}{nat}, \In{s}{infoCliente}}{}
  [$d \igobs d_0$]
  {$d \igobs definir(d_0, c, s)$}
  [$O(\#claves(d) + copy(c) + copy(s))$]
  [define la clave $c$ con el significado $s$ en el diccionario.]
  [los elementos $c$ y $s$ se definen por copia.]
  
  \InterfazFuncion{Obtener}{\In{d}{dictClientes(nat, infoCliente)}, \In{c}{nat} }{infoCliente}
  [def?(c,d)]
  {esAlias($res$, obtener($c$, $d$))}
  [$O(log(n))$]
  [obtiene el significado $\sigma$ que corresponde a la clave $c$.]
  [se genera aliasing entre $res$ y el significado $\sigma$]
  
  ~

  \textbf{Operaciones del iterador}

  \InterfazFuncion{CrearIt}{\In{d}{dicc(nat, infoCliente)}}{itDicc($\alpha$,$\sigma$)}
  [true]
  {alias(esPermutacion(SecuSuby(res), d)) $\land$ vacia?(Anteriores(res))}
  [$\Theta(1)$]
  [crea un iterador unidireccional del diccionario, de forma tal que HayAnterior evalúe a false (i.e., que se pueda recorrer los elementos aplicando iterativamente Siguiente)]
  []
	
 \InterfazFuncion{HaySiguiente}{\In{it}{itDicc(nat, infoCliente)}}{bool}
  [true]
  {res $\igobs$ haySiguiente?(it)}
  [$\Theta(1)$]
  [devuelve true si y sólo si en el iterador todavía quedan elementos para avanzar.]
  []

 \InterfazFuncion{Siguiente}{\In{it}{itDicc(nat,infoCliente)}}{tupla($\alpha$,$\sigma$)}
  [HaySiguiente?(it)]
  {alias(res $\igobs$ Siguiente(it))}
  [$\Theta(1)$]
  [devuelve el elemento siguiente del iterador.]
  [res.significado es modificable si y sólo si it es modificable. En cambio, res.clave no es modificable.]
  
  \InterfazFuncion{SiguienteClave}{\In{it}{itDicc(nat, infoCliente)}}{nat}
  [HaySiguiente?(it)]
  {alias(res $\igobs$ SiguienteClave(it))}
  [$\Theta(1)$]
  [devuelve el elemento siguiente del iterador.]
  [res.significado es modificable si y sólo si it es modificable. En cambio, res.clave no es modificable.]
  
  \InterfazFuncion{SiguienteSignificado}{\In{it}{itDicc(nat, infoCliente)}}{infoCliente}
  [HaySiguiente?(it)]
  {alias(res $\igobs$ SiguienteSignificado(it))}
  [$\Theta(1)$]
  [devuelve el elemento siguiente del iterador.]
  [res.significado es modificable si y sólo si it es modificable. En cambio, res.clave no es modificable.]
  
  \InterfazFuncion{Avanzar}{\In{it}{itDicc(nat, infoCliente)}}{}
  [HaySiguiente?(it)]
  {alias(res $\igobs$ Avanzar(it))}
  [$\Theta(1)$]
  [devuelve el elemento siguiente del iterador.]
  [res.significado es modificable si y sólo si it es modificable. En cambio, res.clave no es modificable.]


\end{Interfaz}

\begin{Representacion}
  
  \textbf{Representación de Diccionario Clientes}

  \begin{Estructura}{dictClientes(nat, infoCliente)}[dc]
    \begin{Tupla}[dc]
      \tupItem{claves}{arreglo(nat)}%
      \tupItem{significados}{arreglo(infoCliente)}%
      \tupItem{tamanio}{nat}%
    \end{Tupla}
  \end{Estructura}
  
  \emph{Invariante de representación}
  
  \begin{itemize}
  	\item La capacidad de los contenedores de claves y significados debe ser la misma.
  	\item dc.tamanio debe indicar la cantidad de entradas en el diccionario y éstas deben estar en las primeras (dc.tamanio-1) primeras posiciones de los respectivos arreglos.
  	\item El arreglo de claves debe estar ordenado.
  \end{itemize}

  \Rep[dc][dc]{(tam($dc$.$claves$)) = tam($d$.$significados$)) $\land$\\
  ($\forall$ $p$, $q$: nat) $p$ < $dc$.$tamanio$ $\impluego$ (definido?($dc$.$claves$, p) $\land$ definido?($dc$.$significados$, p)) $\land$ ($p$ $<$ $q$ $<$ $dc$.$tamanio$ $\Rightarrow$ $dc$.$claves$[$p$] < $dc$.$claves$[$p$])}\mbox{}
  
  ~
 
  \Abs[dictClientes({nat, infoCliente})]{dict(nat, infoCliente)}[dc]{d}{($\forall c:\alpha$)(def?($c$, $d$) $\Leftrightarrow$ $c$ $\in$ arregloAConjunto($dc$.$claves$, tam($dc$.$claves$)) $\land$ (def?($c$, $d$) $\Rightarrow$ obtener($d$, $c$) = $dc$.$significados$[posición($dc$.$claves$, 0, $c$)]))}
  
  ~
  
    \tadOperacion{arregloAConjunto}{arr/arreglo(nat), tamanio/nat}{conj(nat)}{}
  \tadAxioma{arregloAConjunto(arr, tamanio)}{\IF 0?($tamanio$)  THEN $\emptyset$ ELSE {\IF definido?(arr, tamanio-1) THEN Ag(arr[tamanio-1], arregloAConjunto(arr, tamanio-1) ELSE arregloAConjunto(arr, tamanio-1) FI} FI}

~
	\tadOperacion{posicion}{arr/arreglo(nat), pos/nat, buscado/nat}{nat}{buscado $\in$ arregloAConjunto(arr, tam(arr))}
\tadAxioma{posicion(arr, pos, buscado)}{\IF arr[pos] = buscado  THEN pos ELSE posicion(arr, pos+1, buscado) FI}
  
  \BlankLine
  \textbf{Representación del iterador}

  \begin{Estructura}{itDictClientes(nat, infoCliente)}[iter]
    \begin{Tupla}[iter]
      \tupItem{posición}{nat}%
      \tupItem{límite}{nat}%
      \tupItem{claves}{puntero(arrOrd(nat))}%
      \tupItem{significados}{puntero(arr(infoCliente))}%
    \end{Tupla}
  \end{Estructura}

  \Rep[iter][it]{iter.posición < iter.límite}

  ~

  \Abs[iter]{itUni($\alpha$)}[it]{b}{Siguientes($b$) $=$ arreglosASecuDesde(it.posición, it.límite, it.claves, it.significados)}

  ~
 
\tadOperacion{arreglosASecuDesde}{posición/nat, límite/nat, claves/puntero(arrOrd(nat)), significados/puntero(arrOrd(infoCliente))}{bool}{}
\tadAxioma{arreglosASecuDesde(posición, límite, claves, significados)}
{\IF posición = límite THEN 
	<>
  ELSE
  	<claves[posición], significados[posición]> $\bullet$ arreglosASecuDesde(posición+1, límite, claves, significados)
  FI}
~
  

\begin{Algoritmos}
\\
\begin{algorithm}[H]
\textit{i}vacío(\In{n}{nat}) $\longrightarrow$ res: dictClientes(nat, infoCliente)\\
\BlankLine
dc.claves $\leftarrow$ crearArreglo(n)\\
dc.significados $\leftarrow$ crearArreglo(n)\\
dc.tamanio $\leftarrow$ n\\
res $\leftarrow$ dc\\
\end{algorithm}

\begin{algorithm}[H]
\textit{i}definir(\Inout{dc}{dictClientes(nat, infoCliente), \In{c}{nat}, \In{s}{infoCliente}})\\
\BlankLine
posActual $\leftarrow$ dc.tamanio\\
\While{(dc.claves[posActual-1] > c)}{
		dc.claves[posActual] $\leftarrow$ dc.claves[posActual-1]\\
		dc.significados[posActual $\leftarrow$ dc.significados[posActual-1]\\
		posActual--;\\
}
dc.claves[posActual] $\leftarrow$ c\\
dc.significados[posActual] $\leftarrow$ s\\
dc.tamanio += 1\\
\end{algorithm}

\begin{algorithm}[H]
\textit{i}obtener(\Inout{dc}{dictClientes(nat, infoCliente), \In{c}{nat}}) $\longrightarrow$ res: infoCliente\\
\BlankLine
der $\leftarrow$ dc.tamanio-1\\
izq $\leftarrow$ 0\\
medio $\leftarrow$ dc.tamanio/2\\
\While{dc.clave[medio] != c}{
	\If{dc.clave[medio] > c}{
		der $\leftarrow$ medio
	}
	\If{dc.clave[medio] < c}{
		izq $\leftarrow$ medio
	}
}
res $\leftarrow$ dc.significados[medio]\\
\end{algorithm}

\begin{algorithm}[H]
\textit{i}crearIt(\Inout{dc}{dictClientes(nat, infoCliente))} $\longrightarrow$ res: itDictClientes(nat, infoCliente)\\
\BlankLine
it.posicion $\leftarrow$ 0\\
it.limite $\leftarrow$ tamanio(dc.significados)\\
it.claves $\leftarrow$ \&(dc.claves)\\
it.significados $\leftarrow$ \&(dc.significados)\\

res $\leftarrow$ it\\
\end{algorithm}

\begin{algorithm}[H]
\textit{i}haySiguiente(\In{it}{itDictClientes(nat, infoCliente)}) $\longrightarrow$ res: bool\\
\BlankLine
res $\leftarrow$ it.posición < it.límite\\
\end{algorithm}

\begin{algorithm}[H]
\textit{i}siguiente(\Inout{dc}{itDictClientes(nat, infoCliente)}) $\longrightarrow$ res: tupla(nat, infoCliente)\\
\BlankLine
res $\leftarrow$ <*(it.claves[it.posición]), *(it.significados[it.posición])>\\
\end{algorithm}

\begin{algorithm}[H]
\textit{i}siguienteClave(\Inout{dc}{itDictClientes(nat, infoCliente)}) $\longrightarrow$ res: nat\\
\BlankLine
res $\leftarrow$ *(it.claves[it.posición])\\
\end{algorithm}

\begin{algorithm}[H]
\textit{i}siguienteSignificado(\Inout{dc}{itDictClientes(nat, infoCliente)}) $\longrightarrow$ res: infoCliente\\
\BlankLine
res $\leftarrow$ *(it.significados[it.posición])\\
\end{algorithm}

\begin{algorithm}[H]
\textit{i}avanzar(\Inout{dc}{itDictClientes(nat, infoCliente)})\\
\BlankLine
it.posición $\leftarrow$ it.posición + 1\\
\end{algorithm}

\end{Algoritmos}


\end{Representacion}


\end{document}