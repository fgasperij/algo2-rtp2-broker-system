\section{Módulo Diccionario Clientes}


\begin{Interfaz}
  
  \textbf{parámetros formales}\hangindent=2\parindent\\
  \parbox{1.7cm}{\textbf{géneros}} $\alpha$\\
  \parbox[t]{1.7cm}{\textbf{función}}\parbox[t]{\textwidth-2\parindent-1.7cm}{%
    \InterfazFuncion{Copiar}{\In{a}{$\alpha$}}{$\alpha$}
    {$res \igobs a$}
    [$\Theta(copy(a))$]
    [función de copia de $\alpha$'s]
  }

  \textbf{se explica con}: \tadNombre{Diccionario($\alpha$,$\sigma$)}, \tadNombre{Iterador Unidireccional(Tupla($\alpha$,$\sigma$))}.

  \textbf{géneros}: \TipoVariable{dictClientes(nat, infoCliente)}, \TipoVariable{itDictClientes(Tupla($\alpha$,$\sigma$))}.

  \textbf{Operaciones básicas de diccionario ordenado}

  \InterfazFuncion{Vacío}{\In{n}{nat}}{dictClientes(nat, infoCliente)}%
  {$res \igobs vacio$}%
  [$O(n)$]
  [genera un diccionario vacío.]

  \InterfazFuncion{Definir}{\Inout{d}{dict(nat, infoCliente)}, \In{c}{nat}, \In{s}{infoCliente}}{}
  [$d \igobs d_0$]
  {$d \igobs definir(d_0, c, s)$}
  [$O(\#claves(d) + copy(c) + copy(s))$]
  [define la clave $c$ con el significado $s$ en el diccionario.]
  [los elementos $c$ y $s$ se definen por copia.]
  
  \InterfazFuncion{Obtener}{\In{d}{dictClientes(nat, infoCliente)}, \In{c}{nat} }{infoCliente}
  [def?(c,d)]
  {esAlias($res$, obtener($c$, $d$))}
  [$O(log(n))$]
  [obtiene el significado $\sigma$ que corresponde a la clave $c$.]
  [se genera aliasing entre $res$ y el significado $\sigma$]
  
  ~

  \textbf{Operaciones del iterador}

  \InterfazFuncion{CrearIt}{\In{d}{dicc(nat, infoCliente)}}{itDicc($\alpha$,$\sigma$)}
  [true]
  {alias(esPermutacion(SecuSuby(res), d)) $\land$ vacia?(Anteriores(res))}
  [$\Theta(1)$]
  [crea un iterador unidireccional del diccionario, de forma tal que HayAnterior evalúe a false (i.e., que se pueda recorrer los elementos aplicando iterativamente Siguiente)]
  []
	
 \InterfazFuncion{HaySiguiente}{\In{it}{itDicc(nat, infoCliente)}}{bool}
  [true]
  {res $\igobs$ haySiguiente?(it)}
  [$\Theta(1)$]
  [devuelve true si y sólo si en el iterador todavía quedan elementos para avanzar.]
  []

 \InterfazFuncion{Siguiente}{\In{it}{itDicc(nat,infoCliente)}}{tupla($\alpha$,$\sigma$)}
  [HaySiguiente?(it)]
  {alias(res $\igobs$ Siguiente(it))}
  [$\Theta(1)$]
  [devuelve el elemento siguiente del iterador.]
  [res.significado es modificable si y sólo si it es modificable. En cambio, res.clave no es modificable.]
  
  \InterfazFuncion{SiguienteClave}{\In{it}{itDicc(nat, infoCliente)}}{nat}
  [HaySiguiente?(it)]
  {alias(res $\igobs$ SiguienteClave(it))}
  [$\Theta(1)$]
  [devuelve el elemento siguiente del iterador.]
  [res.significado es modificable si y sólo si it es modificable. En cambio, res.clave no es modificable.]
  
  \InterfazFuncion{SiguienteSignificado}{\In{it}{itDicc(nat, infoCliente)}}{infoCliente}
  [HaySiguiente?(it)]
  {alias(res $\igobs$ SiguienteSignificado(it))}
  [$\Theta(1)$]
  [devuelve el elemento siguiente del iterador.]
  [res.significado es modificable si y sólo si it es modificable. En cambio, res.clave no es modificable.]
  
  \InterfazFuncion{Avanzar}{\In{it}{itDicc(nat, infoCliente)}}{}
  [HaySiguiente?(it)]
  {alias(res $\igobs$ Avanzar(it))}
  [$\Theta(1)$]
  [devuelve el elemento siguiente del iterador.]
  [res.significado es modificable si y sólo si it es modificable. En cambio, res.clave no es modificable.]


\end{Interfaz}

\begin{Representacion}
  
  \textbf{Representación de Diccionario Clientes}

  \begin{Estructura}{dictClientes(nat, infoCliente)}[dc]
    \begin{Tupla}[dc]
      \tupItem{claves}{arreglo(nat)}%
      \tupItem{significados}{arreglo(infoCliente)}%
      \tupItem{tamanio}{nat}%
    \end{Tupla}
  \end{Estructura}
  
  \emph{Invariante de representación}
  
  \begin{itemize}
  	\item La capacidad de los contenedores de claves y significados debe ser la misma.
  	\item dc.tamanio debe indicar la cantidad de entradas en el diccionario y éstas deben estar en las primeras (dc.tamanio-1) primeras posiciones de los respectivos arreglos.
  	\item El arreglo de claves debe estar ordenado.
  \end{itemize}

  \Rep[dc][dc]{(tam($dc$.$claves$)) = tam($d$.$significados$)) $\land$\\
  ($\forall$ $p$, $q$: nat) $p$ < $dc$.$tamanio$ $\impluego$ (definido?($dc$.$claves$, p) $\land$ definido?($dc$.$significados$, p)) $\land$ ($p$ $<$ $q$ $<$ $dc$.$tamanio$ $\Rightarrow$ $dc$.$claves$[$p$] < $dc$.$claves$[$p$])}\mbox{}
  
  ~
 
  \Abs[dictClientes({nat, infoCliente})]{dict(nat, infoCliente)}[dc]{d}{($\forall c:\alpha$)(def?($c$, $d$) $\Leftrightarrow$ $c$ $\in$ arregloAConjunto($dc$.$claves$, tam($dc$.$claves$)) $\land$ (def?($c$, $d$) $\Rightarrow$ obtener($d$, $c$) = $dc$.$significados$[posición($dc$.$claves$, 0, $c$)]))}
  
  ~
  
    \tadOperacion{arregloAConjunto}{arr/arreglo(nat), tamanio/nat}{conj(nat)}{}
  \tadAxioma{arregloAConjunto(arr, tamanio)}{\IF 0?($tamanio$)  THEN $\emptyset$ ELSE {\IF definido?(arr, tamanio-1) THEN Ag(arr[tamanio-1], arregloAConjunto(arr, tamanio-1) ELSE arregloAConjunto(arr, tamanio-1) FI} FI}

~
	\tadOperacion{posicion}{arr/arreglo(nat), pos/nat, buscado/nat}{nat}{buscado $\in$ arregloAConjunto(arr, tam(arr))}
\tadAxioma{posicion(arr, pos, buscado)}{\IF arr[pos] = buscado  THEN pos ELSE posicion(arr, pos+1, buscado) FI}
  
  \BlankLine
  \textbf{Representación del iterador}

  \begin{Estructura}{itDictClientes(nat, infoCliente)}[iter]
    \begin{Tupla}[iter]
      \tupItem{posición}{nat}%
      \tupItem{límite}{nat}%
      \tupItem{claves}{puntero(arrOrd(nat))}%
      \tupItem{significados}{puntero(arr(infoCliente))}%
    \end{Tupla}
  \end{Estructura}

  \Rep[iter][it]{iter.posición < iter.límite}

  ~

  \Abs[iter]{itUni($\alpha$)}[it]{b}{Siguientes($b$) $=$ arreglosASecuDesde(it.posición, it.límite, it.claves, it.significados)}

  ~
 
\tadOperacion{arreglosASecuDesde}{posición/nat, límite/nat, claves/puntero(arrOrd(nat)), significados/puntero(arrOrd(infoCliente))}{bool}{}
\tadAxioma{arreglosASecuDesde(posición, límite, claves, significados)}
{\IF posición = límite THEN 
	<>
  ELSE
  	<claves[posición], significados[posición]> $\bullet$ arreglosASecuDesde(posición+1, límite, claves, significados)
  FI}
~
  

\begin{Algoritmos}
\\
\begin{algorithm}[H]
\textit{i}vacío(\In{n}{nat}) $\longrightarrow$ res: dictClientes(nat, infoCliente)\\
\BlankLine
dc.claves $\leftarrow$ crearArreglo(n)\\
dc.significados $\leftarrow$ crearArreglo(n)\\
dc.tamanio $\leftarrow$ n\\
res $\leftarrow$ dc\\
\end{algorithm}

\begin{algorithm}[H]
\textit{i}definir(\Inout{dc}{dictClientes(nat, infoCliente), \In{c}{nat}, \In{s}{infoCliente}})\\
\BlankLine
posActual $\leftarrow$ dc.tamanio\\
\While{(dc.claves[posActual-1] > c)}{
		dc.claves[posActual] $\leftarrow$ dc.claves[posActual-1]\\
		dc.significados[posActual $\leftarrow$ dc.significados[posActual-1]\\
		posActual--;\\
}
dc.claves[posActual] $\leftarrow$ c\\
dc.significados[posActual] $\leftarrow$ s\\
dc.tamanio += 1\\
\end{algorithm}

\begin{algorithm}[H]
\textit{i}obtener(\Inout{dc}{dictClientes(nat, infoCliente), \In{c}{nat}}) $\longrightarrow$ res: infoCliente\\
\BlankLine
der $\leftarrow$ dc.tamanio-1\\
izq $\leftarrow$ 0\\
medio $\leftarrow$ dc.tamanio/2\\
\While{dc.clave[medio] != c}{
	\If{dc.clave[medio] > c}{
		der $\leftarrow$ medio
	}
	\If{dc.clave[medio] < c}{
		izq $\leftarrow$ medio
	}
}
res $\leftarrow$ dc.significados[medio]\\
\end{algorithm}

\begin{algorithm}[H]
\textit{i}crearIt(\Inout{dc}{dictClientes(nat, infoCliente))} $\longrightarrow$ res: itDictClientes(nat, infoCliente)\\
\BlankLine
it.posicion $\leftarrow$ 0\\
it.limite $\leftarrow$ tamanio(dc.significados)\\
it.claves $\leftarrow$ \&(dc.claves)\\
it.significados $\leftarrow$ \&(dc.significados)\\

res $\leftarrow$ it\\
\end{algorithm}

\begin{algorithm}[H]
\textit{i}haySiguiente(\In{it}{itDictClientes(nat, infoCliente)}) $\longrightarrow$ res: bool\\
\BlankLine
res $\leftarrow$ it.posición < it.límite\\
\end{algorithm}

\begin{algorithm}[H]
\textit{i}siguiente(\Inout{dc}{itDictClientes(nat, infoCliente)}) $\longrightarrow$ res: tupla(nat, infoCliente)\\
\BlankLine
res $\leftarrow$ <*(it.claves[it.posición]), *(it.significados[it.posición])>\\
\end{algorithm}

\begin{algorithm}[H]
\textit{i}siguienteClave(\Inout{dc}{itDictClientes(nat, infoCliente)}) $\longrightarrow$ res: nat\\
\BlankLine
res $\leftarrow$ *(it.claves[it.posición])\\
\end{algorithm}

\begin{algorithm}[H]
\textit{i}siguienteSignificado(\Inout{dc}{itDictClientes(nat, infoCliente)}) $\longrightarrow$ res: infoCliente\\
\BlankLine
res $\leftarrow$ *(it.significados[it.posición])\\
\end{algorithm}

\begin{algorithm}[H]
\textit{i}avanzar(\Inout{dc}{itDictClientes(nat, infoCliente)})\\
\BlankLine
it.posición $\leftarrow$ it.posición + 1\\
\end{algorithm}

\end{Algoritmos}


\end{Representacion}
