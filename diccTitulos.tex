\section{Módulo Diccionario Títulos }
%\section{Módulo Diccionario Trie($\alpha$)}
%Estado: En proceso
%---------------------------------------------------------------
%El Tad Títulos se representa con un diccionario de tuplas
%El Diccionario se representa con Trie.
%Clave: nombre del título.
%Significado: Tupla < #maxAcciones, accionesDisponibles, cotización, enAlza >
%#maxAcciones, accionesDisponibles, cotización son Nat.
%enAlza es bool.
%---------------------------------------------------------------
\SetKw{Orden}{Orden Complejidad:}
\begin{Interfaz}
  
  \textbf{parámetros formales}\hangindent=2\parindent\\
  \parbox{1.7cm}{\textbf{géneros}} $\alpha$\\
  \parbox[t]{1.7cm}{\textbf{función}}\parbox[t]{\textwidth-2\parindent-1.7cm}{%
    \InterfazFuncion{Copiar}{\In{a}{$\alpha$}}{$\alpha$}
    {$res \igobs a$}
    [$\Theta(copy(a))$]
    [función de copia de $\alpha$'s]
  }

  \textbf{se explica con}: \tadNombre{Diccionario(String, $\alpha$)}.

  \textbf{géneros}: \TipoVariable{diccTítulo($String$,$\alpha$)}, \TipoVariable{itDicc$(\alpha)$}.

  \Title{Operaciones básicas de diccionario títulos}

  \InterfazFuncion{NuevoDiccionario}{}{diccTítulo($String$,$\alpha$)}%
  {$res \igobs$ vacio()}%
  [$\Theta(1)$]
  [Crea un nuevo diccionario vacío.]
  
  \InterfazFuncion{Definir}{\In{c}{string}, \In{s}{$\alpha$}, \Inout{d}{diccTítulo($String$,$\alpha$)}}{}
  [$d$ $\igobs$ $d_{0}$]
  {$d$ $\igobs$ definir($d_{0}$, $c$, $s$)}
  [$\Theta(|c| + copy(s))$]
  [Define la clave $c$ con el significado $s$ en el diccionario $d$.]
  [Se agrega por copia el significado $s$] 

  \InterfazFuncion{Obtener}{\In{c}{string}, \In{d}{diccTítulo($String$,$\alpha$)}}{$\alpha$}
  [def?($c$, $d$)]
  {$res$ $\igobs$ obtener($c$, $d$)}
  [$\Theta(|c|)$]
  [Devuelve el significado de la clave $c$ contenida en el diccionario $d$.]
  []
  
  \InterfazFuncion{Definido?}{\In{c}{string}, \In{d}{diccTítulo($String$,$\alpha$)}}{$bool$}
  {$res$ $\igobs$ def?($c$, $d$)}
  [$\Theta(|c|)$]
  [Chequea si está definida la clave $c$ en el diccionario $d$.]
  []

  \InterfazFuncion{\#Claves}{\In{d}{diccTítulo($String$,$\alpha$)}}{nat}
  {$res$ $\igobs$ \#(claves($d$))}
  [$\Theta(1)$]
  [Devuelve la cantidad de claves del diccionario.]
  
%----------------------------------------------------------
% Operaciones del Iterador
% Idem a las de diccionario lineal del apunte de modulos básicos
% Iteramos como si fuera un diccionario lineal O(1)
% Buscamos y obtenemos utilizando la estructura trie O(|c|)

  \Title{Operaciones del iterador}
  \InterfazFuncion{CrearIt}{\In{d}{diccTítulo($String$,$\alpha$)}}{itDicc($String$,$\alpha$)}
  {$res$ $\igobs$ < crearItUni(<>,$d$.claves), crearItUni(<>,$d$.alfas)>}
  [$\Theta(1)$]
  [Crea un iterador unidireccional del diccionario. Se pueden recorrer los elementos aplicando iterativamente Siguiente]
  % bidireccional del diccionario, de forma tal que $\NombreFuncion{HayAnterior}$ evalúe a $\ensuremath{\mathit{false}}$ (ie. que se pueda recorrer los elementos aplicando iterativamente $\NombreFuncion{Siguiente}$).]
  
  \InterfazFuncion{HaySiguiente?}{\In{it}{itDicc($String$,$\alpha$)}}{bool}
  {$res$ $\igobs$ haySiguiente?($it$)}
  [$\Theta(1)$]
  [Devuelve true si y sólo si en el iterador todavía quedan elementos para avanzar]
    
  %\InterfazFuncion{Actual}{\In{it}{itDicc$(\alpha)$}}{String}
  %[HaySiguiente?($it$)]
  %{alias($res$ $\igobs$ Actual($it$)}
  %%{alias($res$ $\igobs$ Siguiente($it$))}
  %[$\Theta(1)$]
  %[Devuelve el elemento actual del iterador (tupla: clave - puntero a significado)]
  %%[La clave no es modificable]

  %\InterfazFuncion{ActualClave}{\In{it}{itDicc$(\alpha)$}}{String}
  %[HaySiguiente?($it$)]
  %{alias($res$ $\igobs$ $\prod_1$(Actual($it$))}
  %%{alias($res$ $\igobs$ Siguiente($it$))}
  %[$\Theta(1)$]
  %[Devuelve la clave actual del iterador]
  %%[La clave no es modificable]

  %\InterfazFuncion{ActualSignificado}{\In{it}{itDicc$(\alpha)$}}{$\alpha$}
  %[HaySiguiente?($it$)]
  %{alias($res$ $\igobs$ $\prod_2$(Actual($it$))}
  %%{alias($res$ $\igobs$ Siguiente($it$))}
  %[$\Theta(1)$]
  %[Devuelve el significado actual del iterador]

  \InterfazFuncion{Siguiente}{\In{it}{itDicc($String$,$\alpha$)}}{tupla($String$, $\alpha$)}
  [HaySiguiente?($it$)]
  {alias($res$ $\igobs$ Siguiente($it$))}
  [$\Theta(1)$]
  [Devuelve el elemento siguiente del iterador]
  [$res$.significado es un puntero al objeto $\alpha$ y es modificable si y sólo si $it$ es modificable. En cambio, $res$.clave no es modificable] 

  \InterfazFuncion{SiguienteClave}{\In{it}{itDicc($String$,$\alpha$)}}{String}
  [HaySiguiente?($it$)]
  {alias($res$ $\igobs$ Siguiente($it$).clave)}
  [$\Theta(1)$]
  [Devuelve la clave del elemento siguiente del iterador]
  [$res$ no es modificable] %PREGUNTAR!!!

  \InterfazFuncion{SiguienteSignificado}{\In{it}{itDicc($String$,$\alpha$)}}{$\alpha$}
  [HaySiguiente?($it$)]
  {alias($res$ $\igobs$ Siguiente($it$))}
  [$\Theta(1)$]
  [Devuelve el significado del elemento siguiente del iterador]
  [$res$ es modificable si y sólo si $it$ es modificable]
  
  \InterfazFuncion{Avanzar}{\Inout{it}{itDicc($String$,$\alpha$)}}{}
  [$it = it_0$ $\land$ HaySiguiente?($it$)]
  {$it$ $\igobs$ Avanzar($it_0$)}
  [$\Theta(1)$]
  [Avanza a la posición siguiente del iterador]
  
  
\end{Interfaz}

\begin{Representacion}\\
\tab\textbf{Representación del diccionario:}
  \begin{Estructura}{diccTítulo($String$, $\alpha$)}[estr\_diccTítulo]
    \begin{Tupla}[estr\_diccTítulo]
      \tupItem{raíz}{estr\_nodo}
      \tupItem{claves}{lista(String)}
    \end{Tupla}
      
    \begin{Tupla}[estr\_nodo]
      \tupItem{significado}{puntero $(\alpha)$}
      \tupItem{hijos}{array[256] de puntero(estr\_nodo)}%
    \end{Tupla}
  \end{Estructura}  

 \textbf{Invariante de representación:}
 \begin{enumerate}
 	\item Dos nodos no pueden compartir un hijo
    \item Sin ciclos en el árbol
    \item Las hojas del árbol no pueden tener significado nulo
    \item La cantidad de claves ingresadas en e.claves debe ser igual a la cantidad de significados válidos (distintos de NULL) del árbol Trie
    \item Las claves contenidas en $e$.$claves$ deben estar definidas en el arbol Trie
 \end{enumerate}

\end{Representacion}

\Rep[estr\_diccTítulo][e]{ \\
    SinCiclos($e$.raíz, $\emptyset$) $\yluego$ \hfill 1. \\
    NoCompartenHijos($e$.raíz) $\yluego$ \hfill 1. \\
    SignificadosHojasNotNull($e$.raíz) $\land$ \hfill 2. \\    
    \TipoVariable {Longitud($e$.claves)} $\igobs$ CantSignificados($e$.raíz) $\land$ \hfill 3. \\
    ($\forall clave:String$) ((está?($clave$, $e$.claves)) $\Leftrightarrow$ definido?($clave$,$e$.raíz)  )\hfill 4.\\
  }
 ~

\tadOperacion{SignificadoHojaNotNull}{estr\_nodo/e}{bool}{}
\tadAxioma{SignificadoHojaNotNull($e$)}
{\IF Hoja?($e$, 0) THEN 
	$\neg$(significado($e$)$\igobs$ NULL)
  ELSE
  	RecorrerHijos($e$,0)
  FI}
\tadOperacion{Hoja?}{estr\_nodo/e, nat}{bool}{}
\tadAxioma{Hoja?($e$,$n$)}
{($e$.hijos[$n$] $\igobs$ NULL) $\land$
\IF ($n$ < 256) THEN
	Hoja?($e$,$n$+1)
ELSE
	true
FI}

~

\tadOperacion{RecorrerHijos}{estr\_nodo/e, nat}{bool}{SinCiclos($e$,0)}
\tadAxioma{RecorrerHijos($e$,$n$)}
{(\IF ($\neg$($e$.hijos[$n$] $\igobs$ NULL)) THEN
	SignificadoHojaNotNull(*(e.hijos[$n$]))
  ELSE
   true
  FI) $\land$
	({\IF ($n$ < 256) THEN 
		RecorrerHijos($e$,$n$+1)
	ELSE
		true
	FI)}
FI}

\tadOperacion{CantSignificados}{estr\_nodo/$e$}{nat}{}
\tadAxioma{CantSignificados($e$)}
{(\IF ($e$.significado $\igobs$ NULL) THEN
	0
ELSE
	1
FI) + SigHijos(e,0)}

\tadOperacion{SigHijos}{estr\_nodo/$e$, nat}{nat}{}
\tadAxioma{SigHijos($e$,$n$)}
{(\IF ($e$.hijos[$n$] $\igobs$ NULL) THEN
	0
ELSE
	CantSignificados(*($e$.hijos[$n$]))
FI) + (\IF ($n$ < 256) THEN
	SigHijos($e$,$n$+1)
ELSE
	0
FI)}

%Abstraccion DICTIONARIO
\textbf{Función de abstracción:}
\Abs[estr\_diccTítulo]{dicc(string, $\alpha$)}[e]{d}{
    \#(claves($d$)) $\igobs$ \TipoVariable {Longitud($e$.clave)} $\yluego$ \hfill 1. \\
    ($\forall c$: string)(def?($c$, $d$) $\impluego$  \hfill 2. \\
       $\tab$ (definido?($c$, $e$.raíz) $\yluego$  \\
       $\tab \tab$ obtener($c$, $d$) $\igobs$ *(ObtDeEstruc($c$, $e$.raíz)) )
    )
}

\tadOperacion{definido?}{string/e, estr\_nodo/$e$}{$bool$}{}
\tadAxioma{definido?($c$,$n$)}
{\IF vacía?($c$) THEN
	$\neg$(n.significado = NULL)
ELSE
	{\IF n.hijos[ORD(prim(c)] = NULL THEN false 
    ELSE 
    	definido?(fin($c$),$n$.hijos[ORD(prim($c$))]    
    FI}
FI}
 
\tadOperacion{ObtDeEstruc}{string/$ e$, estr\_nodo/$e$}{$\alpha$}{}
\tadAxioma{ObtDeEstruc($c$,$n$)}
{\IF vacía?($c$) THEN
	n.significado
ELSE		
   	ObtDeEstruc(fin($c$),$n$.hijos[ORD(prim($c$))]    
FI}

%REPresentacion iterador
 \textbf{Representación del iterador:}\\ \tab
 El iterador del diccionario es simplemente un iterador a la lista de claves.\\\tab Lo único que hay que pedir es que satisfaga el Rep de esta lista.\\\tab Por implementación, alcanza con que sea unidireccional. 
 	\begin{Estructura}{itDiccTítulos($String$,$\alpha$)}[itDic]
    \begin{Tupla}[itDic]
		\tupItem{claves}{itLista($String$)}
    \end{Tupla}
 	\end{Estructura}

\Rep[itDic][it]{ \\
	Rep(it.claves)
  }
Abs : itDic $it$ $\longrightarrow$ itUni($String$) \hfill\{Rep($it$)\}\\ \hspace*{4mm}

Abs($it$) $\equiv$ CrearItUni(Siguientes(it.claves)))

%%%%%%%%%%%%%%%%%%%%%%%%%%%%%%%%%%%%%%%%%%%%%%%%%%%%%%%%%%%%%%%%%%%%%%%%%%%%%%%%%%%%%%%%%%%%%%%%%%%%%%%%%%%%%%%%%%%%%%%%%%%%%%%%%%%%%%%%%%%%%%%%%%
%ALGORITMOS
\begin{Algoritmos}
 \incmargin{1em}
    %\linesnumbered
   %\restylealgo{boxed}
    \dontprintsemicolon
\\
%NuevoDiccionario
\tab\textit{i}NuevoDiccionario() $\longrightarrow$ res: bool\\
\hspace*{9mm}\Orden{O(1)}\\
\begin{algorithm}[H]
%\BlankLine
	res $\leftarrow$ <raíz: iNodoNuevo(), Claves : Vacía(), Alfas : Vacía() > \tcp*{O(1)}
\end{algorithm}

%Nodo Nuevo
\textit{i}NodoNuevo() $\longrightarrow$ res: estr\_nodo\\
\hspace*{9mm}\Orden{O(1)}\\
\begin{algorithm}[H]
%\BlankLine
	res $\leftarrow$ <significado: NULL, hijos:CrearArreglo() > \tcp*{O(1)}
	\For{var $i : nat\leftarrow 0$ \KwTo $255$\tcp*{O(256)}}{
		res.hijos[i] $\leftarrow$ NULL 
	}
\end{algorithm}

%Definir
\textit{i}Definir(\In{c}{string}, \In{s}{$\alpha$}, \Inout{d}{estr\_diccTítulo}) $\longrightarrow$ res: estr\_dicTrie\\
\hspace*{9mm}\Orden{O(|c|)}\\
\begin{algorithm}[H]
	var $actual$ : \TipoVariable{puntero($estr\_nodo$)} $\leftarrow$ \&($d$.raíz) \tcp*{O( 1 )}
	\For{var $i : nat\leftarrow 0$ \KwTo (Longitud($c$) - 1)\tcp*{O(|c|)}}{ 
		\If{actual $\rightarrow$ hijos[ORD(c[i])] = NULL}{
				AgregarAtrás(d.$Claves$, c) \tcp*{O(|c|)}
			actual $\rightarrow$ hijos[ORD(c[i])] $\leftarrow$ $\&$(iNodoNuevo())
		}
		actual $\leftarrow$ (actual $\rightarrow$ hijos[ORD(c[i])])
	}
	actual $\rightarrow$ significado $\leftarrow \&$(Copiar($s$))\tcp*{O( 1 )} 
\end{algorithm}

%Obtener
\textit{i}Obtener(\In{c}{string}, \In{d}{estr\_diccTítulo}) $\longrightarrow$ res:$\alpha$ \\
\hspace*{9mm}\Orden{O(|c|)}\\
\begin{algorithm}[H]
%\BlankLine
	var $actual$ : \TipoVariable{puntero($estr\_nodo$)} $\leftarrow$ \&($d$.raíz)\tcp*{O( 1 )}
	\For{var $i : nat\leftarrow 0$ \KwTo (Longitud($c$) - 1)\tcp*{O(|c|)}}{
		actual $\leftarrow$ (actual $\rightarrow$ hijos[ORD(c[i])])
	}
	res $\leftarrow$ * (actual $\rightarrow$ significado)\tcp*{O( 1 )} 
\end{algorithm}

%Definido
% Ejemplo: Si esta definido CASA
% Casos: 
%	1 - puede estar definido CASAS pero no CASA -> Significado en CASA = NULL al salir del while
%	2 - puede estar definido CAS o algo menor a CASA:  actual -> NULL al salir del while
%	3 - pueden estar definido CASA y CASAS: sale del while por i < longitud(c), no podemos chequear solo que actual != NULL 
%	4 - No puedo pedir actual->significado sin antes verificar que actual no apunte a NULL 

\textit{i}Definido?(\In{c}{string}, \In{d}{estr\_diccTítulo})$\longrightarrow$ res: bool \\
\hspace*{9mm}\Orden{O(|c|)}\\
\begin{algorithm}[H]
%\BlankLine
	var $actual$ : \TipoVariable{puntero($estr\_nodo$)} $\leftarrow$ \&($d$.raíz) \tcp*{O( 1 )}
	var i : \TipoVariable{Nat} $\leftarrow$ $0$ \tcp*{O( 1 )}
	\While{ $actual$ != NULL $\&\&$  ($i$ < Longitud($c$))\tcp*{O(|c|)}}{
		$i$ $\leftarrow$ $i$ + 1 \\
		actual $\leftarrow$ (actual $\rightarrow$ hijos[ORD(c[i])])
	}
	\eIf{($actual$ != NULL)}{
		\eIf{($actual$ $\rightarrow$ significado) != NULL}{
			res $\leftarrow$ $true$\tcp*{O( 1 )}
		}{
			res $\leftarrow$ $false$\tcp*{O( 1 )}
		}
	}{
		res $\leftarrow$ $false$\tcp*{O( 1 )}
	}

\end{algorithm}

%Cantidad de Claves
\textit{i}$\#$Claves(\In{d}{estr\_diccTítulo})$\longrightarrow$ res: nat\\
\hspace*{9mm}\Orden{O(1)}\\
\begin{algorithm}[H]
%\BlankLine
	res $\leftarrow$ Longitud($d$.Claves)\tcp*{O( 1 )}
\end{algorithm}

\textbf {Algoritmos del iterador}

\textit{i}CrearIT(in d: diccTítulo($String$,$\alpha$)) $\longrightarrow$ res: ItDicc($String$,$\alpha$)\\
\hspace*{9mm}\Orden{O(1)}\\
\begin{algorithm}[H]
\BlankLine
res $\leftarrow$ claves:CrearIt($d$.claves)
\end{algorithm}

\textit{i}HaySiguiente?(in it: ItDicc($String$,$\alpha$)) $\longrightarrow$ res: bool\\
\hspace*{9mm}\Orden{O(1)}\\
\begin{algorithm}[H]
\BlankLine
res $\leftarrow$ HaySiguiente(it.claves)
\end{algorithm}

\textit{i}Siguiente(in it: ItDicc($String$,$\alpha$)) $\longrightarrow$ res: $String$\\
\hspace*{9mm}\Orden{O(1)}\\
\begin{algorithm}[H]
\BlankLine
res $\leftarrow$ Siguiente(it.claves)
\end{algorithm}

\textit{i}Avanzar(in it: ItDicc($String$,$\alpha$)) $\longrightarrow$  res: ItDicc($String$)\\
\hspace*{9mm}\Orden{O(1)}\\
\begin{algorithm}[H]
\BlankLine
res $\leftarrow$ Avanzar(it.claves)
\end{algorithm}

\end{Algoritmos}
